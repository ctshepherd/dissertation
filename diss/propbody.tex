\vfil

\centerline{\Large Computer Science Project Proposal}
\vspace{0.4in}
\centerline{\Large PFS: A Distributed Peer-to-Peer Filesystem based on the Paxos protocol}
\vspace{0.4in}
\centerline{\large Charlie Shepherd, Churchill College}
\vspace{0.3in}
\centerline{\large Originator: Charlie Shepherd}
\vspace{0.3in}
\centerline{\large 5$^{th}$ October 2012}

\vfil


\noindent
{\bf Project Supervisor:} Stephen Cross
\vspace{0.2in}

\noindent
{\bf Director of Studies:} Dr John Fawcett
\vspace{0.2in}
\noindent

\noindent
{\bf Project Overseers:} Dr~A.~Madhavapeddy \& Dr~M.~Kuhn


% Main document

\section*{Introduction, The Problem To Be Addressed}

Talk about filesystems, bitorrent, dropbox, etc
modernisation, smart phones, tablets, laptops

There has been an explosion of connectivity recently, with each consumer owning multiple
laptops, a tablet, smartphone and a desktop computer. The demand for services that provide the
ability to synchronise a users data across their multiple devices has grown accordingly, the rise
to prominence of Dropbox being the most pertinent recent example (quote valuation of dropbox).

Some others include...

Lots of synchronisation applications are now moving to the cloud, eg Dropbox. However, there are
several drawbacks to this approach:

Lack of control of data

If a users data resides on their synchronisation provider's system, then if that provider goes
bust, suffers a natural or manmade disaster or their equipment breaks, the user may be deprived of
their ability to use the service, and may in fact lose all their data completely. Network outages
to the region where the data is stored (Uganda, AWS outage in US), may also deny the user access
to their data, or to their service, even if they are in a different region or even country.

Clearly distributed networks are far more resilient to failure, both hardware and network.

- Lack of privacy of data
All the users data, along with connection and login information, can be collected by the central
repository and retained indefinitely. In fact many regulations require the retention of this
information by service providers. In the past this information has been subpoenaed by governments,
including the United States government, in order to trace an individuals identity, often without
a warrant. Clearly this could be a significant concern to anyone wanting to synchronise sensitive
files, which may include bank details, passwords/PINS, photos etc.

Also, a service provider is free to, and in some cases required by legislation to, inspect the
contents of users data. Although some services like Amazon's S3 enforce encryption so that it is
not possible to snoop on a users data, there have been various recent controversies that show that
this is not a theoretical problem. Although customer level encryption can mitigate this problem,
it is often bulky, complex, not integrated with the synchronisation solution and easy to get
wrong, completely bypassing its benefits.

- reliance on other backup solutions
As there is a master central server, this contains all revision history, data etc and is a
centralised backup source. The advantage of this is that it does provide a centralised location to
backup from, however as this is in the cloud a user has far less control over the ability to
backup. This could be mitigated by hosting the service on the users own service but a distributed
approach is another solution to the problem, which has the advantage of placing the backup
controller as merely another peer in the network, allowing separation of duties between servers
and clearer failover characteristics.

From this it is clear that using the cloud as a synchronisation backend has advantages but also
some clear limitations. For this reason I am proposing to implement a filesystem for file
synchronisation, that uses a peer to peer protocol.


\section*{Starting Point}

{\em Describe existing state of the art, previous work in this area, libraries and databases to be used.
Describe the state of any existing codebase that is to be built on.  }


Google uses the Paxos algorithm in their Chubby distributed lock service in order to keep
replicas consistent in case of failure. Chubby is used by BigTable which is now in production
in Google Analytics and other products.
Google Spanner uses the Paxos algorithm internally
The OpenReplica replication service uses Paxos to maintain replicas for an open access
system that enables users to create fault-tolerant objects. It provides high
performance through concurrent rounds and flexibility through dynamic membership
changes.
Microsoft uses Paxos in the Autopilot cluster management service from Bing.
WANDisco have implemented Paxos within their DConE active-active
replication technology.


\section*{Resources Required}

{\em A note of the resources required and confirmation of access.}

I will mainly do my project on a virtual machine which is running on my own personal computer.
Although the source code will
For this project I shall mainly use my own quad-core computer that runs Fedora Linux. Backup
will be to github and/or to an SVN repository on an external hard disk that is dumped to writable CD/DVD media.
I have another similar computer to hand should my main machine suddenly fail.
I require no other special resources.

\section*{Work to be done}

{\em Describe the technical work.}

The project breaks down into the following sub-projects:

\begin{enumerate}

\item A study of network filesystems etc

\item A study of FUSE? (Really?)

\item Developing a prototype fuse plugin in python that implements my protocol

\item Analysing several different performance metrics of the protocol

\item Looking at way to optimise the plugin and perhaps the protocol? To make it more efficient
(solely in terms of performance)

\end{enumerate}

\section*{Success Criterion for the Main Result}


In order for the project to be a success, the project must be:
\begin{enumerate}

\item Preserving consistency etc

\end{enumerate}




\section*{Possible Extensions}

{\em Potential further envisaged evaluation metrics or extensions.}

Evaluate:
As for evaluation, in terms of performance you'll probably want to assess
things like latency (how long it takes for a single request to be fulfilled)
and throughput (how many requests you can fulfil per unit time). There's also
failure tolerance (how many nodes can fail before the system breaks), and
performance under failure (how are latency/throughput affected if a certain
number of nodes fail).

Perhaps the system can reconfigure itself to handle node failure; how long does
this take? Also, how does the system scale? What's the performance with ten
nodes vs a hundred? How many requests (from different nodes) can the system
handle before performance degrades?

Are there any cases where the system performs badly? And where it performs
well? How often do the worst case situations occur?

Finally, how resistant is the system to malicious nodes? (This may not be
relevant.)

Security? Encryption?

\section*{Timetable: Workplan and Milestones to be achieved.}


{\em Perhaps list ten or so two-week work-packages.}

Planned starting date is 16/10/2011.

\begin{enumerate}

\item {\bf Michaelmas weeks 2-3} Write initial Paxos functions, functions for debugging etc,
achieve consistency on a decision

\item {\bf Michaelmas weeks 4-5} Build commandline shim, transfer files between nodes, node
discovery etc

\item {\bf Michaelmas weeks 6-7} Enhance protocol support enough for directories and link FUSE
operations class

\item {\bf Michaelmas weeks 8} Enhance protocol support enough for directories and link FUSE
operations class

\item {\bf Michaelmas vacation} Finish coding: test debug etc etc

\item {\bf Lent weeks 0-1} Write progress report. Start analysis of individual nodes/local nodes,
protocol performance/limitations etc

\item {\bf Lent weeks 2-3} Run main experiments, including large scale network across CL, analyse
performance, create graphs in Matlab etc etc

\item {\bf Lent weeks 4-5} Protocol improvement

\item {\bf Lent weeks 6-7} C extension.

\item {\bf Lent weeks 8} Finish evaluation, overarching themes

\item {\bf Easter vacation:} Analyse performance increases etc

\item {\bf Easter term 0-2:}  Further evaluation and complete dissertation.

\item {\bf Easter term 3:} Proof reading and then an early submission so as to concentrate on examination revision.

\end{enumerate}

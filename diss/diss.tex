% Based on template by MR

\documentclass[12pt,twoside,notitlepage]{report}

\usepackage{a4}
\usepackage{verbatim}

\input{epsf}                            % to allow postscript inclusions
% On thor and CUS read top of file:
%     /opt/TeX/lib/texmf/tex/dvips/epsf.sty
% On CL machines read:
%     /usr/lib/tex/macros/dvips/epsf.tex



\raggedbottom                           % try to avoid widows and orphans
\sloppy
\clubpenalty1000%
\widowpenalty1000%

\addtolength{\oddsidemargin}{6mm}       % adjust margins
\addtolength{\evensidemargin}{-8mm}

\renewcommand{\baselinestretch}{1.1}    % adjust line spacing to make
                                        % more readable

\begin{document}

\bibliographystyle{plain}


%%%%%%%%%%%%%%%%%%%%%%%%%%%%%%%%%%%%%%%%%%%%%%%%%%%%%%%%%%%%%%%%%%%%%%%%
% Title


\pagestyle{empty}

\hfill{\LARGE \bf Charlie Shepherd}

\vspace*{60mm}
\begin{center}
\Huge
{\bf PDB: A Distributed Database Based on Paxos} \\
\vspace*{5mm}
Computer Science Tripos \\
\vspace*{5mm}
Churchill College \\
\vspace*{5mm}
\today  % today's date
\end{center}

\cleardoublepage

%%%%%%%%%%%%%%%%%%%%%%%%%%%%%%%%%%%%%%%%%%%%%%%%%%%%%%%%%%%%%%%%%%%%%%%%%%%%%%
% Proforma, table of contents and list of figures

\setcounter{page}{1}
\pagenumbering{roman}
\pagestyle{plain}

\chapter*{Proforma}

{\large
\begin{tabular}{ll}
Name:               & \bf Charlie Shepherd                        \\
College:            & \bf Churchill College                     \\
Project Title:      & \bf PDB: A Distributed Database Based on Paxos \\
Examination:        & \bf Computer Science Tripos, July 2013        \\
Word Count:         & \bf wordcount \\
Project Originator: & Charlie Shepherd                    \\
Supervisor:         & Stephen Cross                    \\
\end{tabular}
}


\section*{Original Aims of the Project}

\subsection*{Project aims}

I aim to implement a distributed database. This will be based on the Paxos protocol.

\subsubsection*{Paxos}

The first half of the project is to implement the Paxos protocol. This will be done in a module,
providing an interface which the database can then use to 

The project must correctly implement the Paxos protocol.
The library must be capable of forming a running network,
in particular dynamic leader election,
as well as achieving consensus on a key/value store across the network.

\subsubsection*{Database}

The database must implement a subset of SQL, specifically:
\begin{enumerate}
\item A single table with a static name
\item SELECT/INSERT
\item WHERE
\item GROUP BY
\item ORDER BY
\item Aggregation
\end{enumerate}

The database must have all ACID properties, that is:


\section*{Work Completed}

All that has been completed appears in this dissertation.

\section*{Special Difficulties}

None.

\newpage
\section*{Declaration}

I, Charlie Shepherd of Churchill College, being a candidate for Part II of the Computer Science
Tripos, hereby declare that this dissertation and the work described in it are my own work,
unaided except as may be specified below, and that the dissertation does not contain material that
has already been used to any substantial extent for a comparable purpose.

\bigskip
\leftline{Signed: }

\medskip
\leftline{Date: \today}

\cleardoublepage

\tableofcontents

\listoffigures

\newpage
\section*{Acknowledgements}

Acknowledge various people

%%%%%%%%%%%%%%%%%%%%%%%%%%%%%%%%%%%%%%%%%%%%%%%%%%%%%%%%%%%%%%%%%%%%%%%
% now for the chapters

\cleardoublepage        % just to make sure before the page numbering
                        % is changed

\setcounter{page}{1}
\pagenumbering{arabic}
\pagestyle{headings}

\chapter{Introduction}

\section{Overview}

\section{Motivation}

I set out to build a distributed database based on Paxos. Paxos was first described in Leslie
Lamport's paper \emph{The Part Time Parliament} \cite{lamport98}.

how it integrates with ecosystem (aka what's already out there).


talk about distributed, p2p advantages


\section{Distributed Consensus Problem}


Paxos vs 2PC etc.

\section{Databases}

\subsection{ACID}

Commonly all databases provide ACID properties. ACID stands for:

\begin{itemize}
\item Atomic
\item Consistent
\item Isolated
\item Durable
\end{itemize}

\subsubsection*{Atomic}

Atomicity means that either an operation completes, or it does not, ie, that the database is not
left in a "halfway" state. This means that we do not need to worry about cleaning up after an
operation, if it does not succeed we can simply retry it, without needing to worry about the state
it has left the database in. Atomicity also applies to transactions in the same way - if we have
a transaction as an operation composed of smaller single operations (for example, INCR B, READ A
-> X, WRITE X+10->A), we don't want some of the operations to complete and some not to. In this
example, we may have the constraint A=10*B. If B was incremented but then the transaction failed
before A was updated, we would leave the database in an inconsistent state.

\subsubsection*{Consistent}

Consistency is very closely related to atomicity, as consistency refers to a property of the
database, and atomicity refers to a property of operations performed on the database. In the
example used for atomicity, we had a constraint on the database (that A=B*10). We want operations
(or transactions) to transform the database from one consistent state to another. This becomes
more pertinent in distributed databases, as we may receive operations in one order at one node,
and in a different order at another node. If we apply the operations in the order that we receive
them, the two nodes are likely to be in inconsistent states. XXX: define consistency in a
distributed system.

\subsubsection*{Isolated}

Isolation means 

Why do we want it?

\subsection{Centralisation vs Distributed}

Centralised databases are used in different situations to distributed databases.
C: Easier to perform organise, edit, query, backup. May slow down under load.
easier to maintain integrity of data
D: Can be slow accessing non local data. Need to ensure consistency

centralised: simpler, faster vs distributed: resilient, scalable


\subsection{Existing Paxos databases}

\section{Summary of chapters}

\cleardoublepage

\chapter{Preparation}

\section{Paxos}

\subsection{Introduction}

Paxos is a distributed consensus protocol. It was developed by Leslie Lamport at Microsoft
research when he was trying to disprove its existence. Paxos is failure tolerant for up to F
simulatneous failures in a network of 2F+1 nodes.

Paxos is actually a family of protocols, based around the same main algorithm. the Paxos algorithm
is an algorithm for agreeing on a single value across a network of processors. Paxos provides
three guarantees: safety, liveness and non-triviality.

\begin{enumerate}
	\item Safety: All nodes eventually agree on the value chosen
	\item Liveness: If a value is proposed, it will eventually be chosen (?? XXX)
	\item Non-triviality: Some value is learnt (this constraint is because the previous two guarantees can
be satisfied without any values being learnt).
\end{enumerate}

Paxos can tolerate certain kinds of failures. These are: messages being delivered late or not at
all, etc XXX.

However Paxos cannot tolerate "rogue" processes, that is, processes deliberately sending malicious
or incorrect messages. There is a variant of Paxos called Byzantine Paxos which can tolerate this,
albeit at a failure tolerance of F for XXX nodes.


\subsection{Definitions}

\subsubsection*{Proposal Numbering}

One of the assumptions that Paxos makes is that every proposal has a unique proposal number. This
is necessary so that proposals have a \emph{total order}, ie we can compare any two proposals to
find the maximum ordered proposal. The conventional way to achieve this is to define a proposal
number as a 2-tuple of (sequence number, node address). These can be compared lexicographically,
and as node addresses are unique, every proposal number will be unique. In practice I plan to use
UUIDs as node identifiers, in order to be confident on uniqueness.

\subsubsection*{Quorums}

Paxos relies on quorums to ensure that 

\subsubsection*{Messages}

Paxos utilises several message types

\subsection{How it works}

\subsection{MultiPaxos}

MultiPaxos is multiple rounds of Paxos occuring at the same time. The way this is outlined in
\emph{Paxos Made Simple} is by a form of leader election. I will outline this here but, for
simplicity, in my project I will simply use multiple Paxos rounds to form the basis of a
distributed operation log. This will be explained in more detail in the Implementation chapter.

\section{ACID}

ACID in more detail


\section{Software Engineering}

\subsection{Module Dependencies}

I chose Twisted in order to make implementing the protocol easier. Twisted provides a lot of
support for implementing protocols and helper classes etc. Also as I am using a state machine
approach to implementing DBP and Paxos, twisted's asynchronous system works very well with this
approach.

\subsection{Programming Language}

There were several options for which programming language to use for this project. C - too
verbose, erlang - too weird, python - just right?

\section{Requirements Analysis}

Requirements for my system:
Transactions
ACID properties
\begin{enumerate}
\item Atomic
\item Consistent
\item Isolated
\item Durable
\end{enumerate}

\subsection{Software Development Process}

Considered using the waterfall model, but didn't fit my requirements (fast prototyping to
understand Paxos). Decided on spiral because of xyz.

\subsection{Version Control and Backup Strategy}

For version control I decided to use Git, as it is a system I am familiar with, and serves my need
both as a VCS and as a remote backup. In my experience it is more usable than other DVCSs such as
Bazaar, Darcs or Mercurial, and much faster than centralised VCSs such as CVS or SVN. Using Git I
backed up my project both to bitbucket, an online repository service which provides free private
repository hosting, and to my own server.

For backups I use the PWF to develop my project on. As it is stored in Git I can back it up simply
by pushing the repo to other hosts. I currently have it backed up to two other hosts.

\subsection{Testing}

Testing is really important
network
lots of effects
lots of subsystems

\subsubsection{Unit Testing}

unit testing really key
regression testing
testing different subsystems

\subsubsection{Integration Testing}

test program etc
see simple change propagate across complex system



\cleardoublepage
\chapter{Implementation}

SQL parser
- what it supports

-start up costs
  - ping time etc
  - inefficiencies
  - cf. "supernodes" vs DHTs to organise nodes

Talk about laptop breaking


\cleardoublepage
\chapter{Evaluation}

This is where the second most amount of marks are gained.

\section{Testing}

\begin{enumerate}
	\item unit test inertia
	\item test programs, see complex effects of single change.
	\item durable - network - stable storage
\end{enumerate}



\cleardoublepage
\chapter{Conclusion}

Conclude here.




\cleardoublepage

%%%%%%%%%%%%%%%%%%%%%%%%%%%%%%%%%%%%%%%%%%%%%%%%%%%%%%%%%%%%%%%%%%%%%
% the bibliography

\addcontentsline{toc}{chapter}{Bibliography}
\bibliography{refs}
\cleardoublepage

%%%%%%%%%%%%%%%%%%%%%%%%%%%%%%%%%%%%%%%%%%%%%%%%%%%%%%%%%%%%%%%%%%%%%
% the appendices
\appendix

\chapter{Project Proposal}

\vfil

\centerline{\Large Computer Science Project Proposal}
\vspace{0.4in}
\centerline{\Large PDB: A Distributed Database Based on Paxos}
\vspace{0.4in}
\centerline{\large Charlie Shepherd, Churchill College}
\vspace{0.3in}
\centerline{\large Originator: Charlie Shepherd}
\vspace{0.3in}
\centerline{\large 18$^{th}$ October 2012}

\vfil


\noindent
{\bf Project Supervisor:} Stephen Cross
\vspace{0.2in}

\noindent
{\bf Director of Studies:} Dr John Fawcett
\vspace{0.2in}
\noindent

\noindent
{\bf Project Overseers:} Dr~A.~Madhavapeddy \& Dr~M.~Kuhn


% Main document

\section*{Introduction, The Problem To Be Addressed}

{\em Paxos} is a protocol for achieving distributed consensus, developed by Leslie Lamport in 1991.

The motivation for Paxos as a protocol is that it is capable of handling failures that other
consensus protocols cannot. {\em Two Phase Commit} (2PC) and {\em Three Phase Commit} (3PC) are two common
protocols that can be used to ensure atomic commits in a distributed system.

2PC works by having a co-ordinator node contact every node and send a proposal message. Each node
must then either respond with a commit or abort message. However, 2PC suffers from several
problems, mainly that it is a blocking protocol. This means that if the co-ordinator fails, and
then a node subsequently fails, the network will deadlock, as 2PC is not able to recover from that
failure situation.

3PC is an extension to 2PC which endeavours to fix this limitation, at the expense of greater
latency, by adding a third roundtrip to confirm the commit to all nodes. This means that the
protocol is asynchronous, and that node failures cannot block the protocol or cause it to fail.
However, it still has its own limitations, in particular, in the event of a network partition. If
the network is partitioned so that in one partition all nodes vote ``commit'' and in the other all
nodes vote ``abort'' both partitions will initiate recovery, and when the network merges again the
system will be in an inconsistent state. This is the limitation that Paxos was intended to solve.

My project will be to design and implement a distributed database, built on the Paxos protocol.


\section*{Starting Point}

My starting point will be to study the Paxos protocol, as well as research distributed databases.
From there I will develop a library implementing Paxos, along with unit tests. I will then design
and implement a distributed database on the Paxos library. The challenge will be to implement a
complex distributed protocol and then utilise it for a database, as these are both areas I have
little experience of.

\section*{Resources Required}

I will mainly do my project on a virtual machine which is running on my own personal laptop.
The source code will be committed to a Git repository, which will be pushed to Bitbucket and my
own personal host. The virtual machine contents will be backed up on an external HDD for quick
recovery, although the git repository will be adequate for restoring my project if the system I am
developing it on fails.
I require no other special resources.

\section*{Work to be done}

The project breaks down into the following sub-projects:

\begin{enumerate}

\item A study of distributed algorithms and the Paxos protocol

\item A study of distributed databases

\item Implementing the Paxos protocol

\item Designing the distributed database

\item Implementing the distributed database

\item Evaluating the performance of the database

\end{enumerate}

\section*{Success Criteria for the Main Result}

In order for the project to be a success, the following must be true:

\begin{enumerate}

\item The project must correctly implement the Paxos protocol.
	The library must be capable of forming a running network,
	in particular dynamic leader election,
	as well as achieving consensus on a key/value store across the network.

\item The database must implement a subset of SQL, specifically:
\begin{enumerate}
\item A single table with a static name
\item SELECT/INSERT
\item WHERE
\item GROUP BY
\item ORDER BY
\item Aggregation
\end{enumerate}

\item The database must have all ACID properties, that is:
\begin{enumerate}
\item Atomic
\item Consistent
\item Isolated
\item Durable
\end{enumerate}

\end{enumerate}

\section*{Evaluation Topics}

There are several potential evaluation metrics for the project.

One major metric is transaction latency - the time for a transaction to be committed to the system. This
can be evaluated in a number of difference circumstances, including simulated node failure, leader
failure and network partition, and the results analysed to see how the system handles performance
under failure compared to normal conditions.

Another key metric is transaction throughput - the maximum number of transactions committed to the
network over a specified period of time. Again there are a number of different situations
throughput can be measured in, including load from one source, load from multiple sources and
load under failure.

I will also investigate the advantages of a distributed database over a normal single-server
database, particularly in terms of scalability. I will also consider how performance is affected
by the ratio of writing clients to reading clients.

\section*{Possible Extensions}

A clear possible extension is to investigate various different modifications to Paxos in order
to try to optimise the database for certain performance characteristics
(e.g. fast reads, but slow writes),
and to assess the usefulness and efficiency of these modifications.

Another possible extension is to investigate how the database performs when various ACID
properties are relaxed, and to measure and analyse how the performance gains compare with the
tradeoffs made.


\section*{Timetable: Workplan and Milestones to be achieved.}

\setlength\parindent{0pt}
\parskip = \baselineskip

Planned starting date is 19/10/2011.

\subsection*{Michaelmas Term}

{\bf 19/10/2012-01/11/2012} Research distributed algorithms and the Paxos protocol; design the
protocol implementation and library layout.

Milestone: A write up of the Paxos algorithm and a design document of the implementation.

{\bf 02/11/2012-15/11/2012} Begin the protocol implementation.

Milestone: Basic Paxos implementation.

{\bf 16/11/2012-29/11/2012} Finish implementation of Paxos library.

Milestone: Paxos implementation that can coordinate distributed leader election and achieve
consensus, including unit tests.

\subsection*{Christmas Vacation}

{\bf 30/11/2012-13/12/2012} Research distributed databases and design the database implementation.

Milestone: A write up of research on distributed databases and a design document for the database.

{\bf 14/12/2012-27/12/2012} Slack time/Revision/Holiday break.

{\bf 28/12/2012-10/01/2013} Prepare for progress report, start database implementation.

Milestone: Draft progress report, initial database implementation.

\subsection*{Lent Term}
{\bf 11/01/2013-24/01/2013} Write progress report, finish database implementation.

Milestone: Finished Progress report, fully functional database implementation.
Deadlines: Progress report deadline - 01/02/2013.

{\bf 25/01/2013-07/02/2013} Perform initial performance analysis on transaction, including
transaction latency and transaction throughput.

Milestone: Initial analysis data.

{\bf 08/02/2013-21/02/2013} Perform detailed performance analysis comparing distributed and
centralised servers, and on failing and partitioned networks.

Milestone: Analysis data on server models and on performance during failure.

{\bf 22/02/2013-07/03/2013} Investigate improvements to the protocol/implementation and their effect on
performance metrics.

Milestone: Improvements to protocol/implementation and revised performance data.

{\bf 08/03/2013-21/03/2013} Start dissertation.

Milestone: Draft Introduction and Preparation sections complete.

\subsection*{Easter Vacation}

{\bf 22/03/2013-04/04/2013} Finish writing up dissertation.

Milestone: Draft Implementation, Evaluation and Conclusion sections complete.

{\bf 05/04/2013-18/04/2013} Proof reading and then an early submission so as to concentrate on
examination revision.

Milestone: Finished dissertation.

\subsection*{Easter Term}
{\bf 19/04/2013-02/05/2013} Slack time/Revision/Holiday break.

{\bf 03/05/2013-16/05/2013} Slack time/Revision/Holiday break.

Deadlines: Official dissertation submission deadline - 17/05/2013.


\end{document}

% Based on template by MR

\documentclass[12pt,twoside,notitlepage]{report}

\usepackage{a4}
\usepackage{verbatim}
\usepackage{natbib}
\usepackage{graphicx}
\usepackage{algorithm}
\usepackage{bm}

\input{epsf}                            % to allow postscript inclusions

\raggedbottom                           % try to avoid widows and orphans
\sloppy
\clubpenalty1000%
\widowpenalty1000%

\addtolength{\oddsidemargin}{6mm}       % adjust margins
\addtolength{\evensidemargin}{-8mm}

\renewcommand{\baselinestretch}{1.1}    % adjust line spacing to make
                                        % more readable

% Resize over-large graphics ( http://tex.stackexchange.com/questions/27083/can-i-conditionally-scale-an-image-with-graphicx )
\newcommand{\lwincludegraphics}[2][]{%
  \sbox{0}{\includegraphics[#1]{#2}}%
  \ifdim\wd0>\linewidth
    \resizebox{\linewidth}{!}{\box0 }%
  \else
    \leavevmode\box0
  \fi}

\newcommand{\msg}[1] {{\bf #1}}         % \msg command for formatting Paxos messages
\newcommand{\op}[1]  {{\bf #1}}         % \op  command for formatting database operations


\begin{document}

\bibliographystyle{plain}


%%%%%%%%%%%%%%%%%%%%%%%%%%%%%%%%%%%%%%%%%%%%%%%%%%%%%%%%%%%%%%%%%%%%%%%%
% Title


\pagestyle{empty}

\hfill{\LARGE \bf Charlie Shepherd}

\vspace*{60mm}
\begin{center}
\Huge
{\bf PDB: A Distributed Database Based on Paxos} \\
\vspace*{5mm}
Computer Science Tripos \\
\vspace*{5mm}
Churchill College \\
\vspace*{5mm}
\today  % today's date
\end{center}

\cleardoublepage

%%%%%%%%%%%%%%%%%%%%%%%%%%%%%%%%%%%%%%%%%%%%%%%%%%%%%%%%%%%%%%%%%%%%%%%%%%%%%%
% Proforma, table of contents and list of figures

\setcounter{page}{1}
\pagenumbering{roman}
\pagestyle{plain}

\chapter*{Proforma}

{\large
\begin{tabular}{ll}
Name:               & \bf Charlie Shepherd                        \\
College:            & \bf Churchill College                     \\
Project Title:      & \bf PDB: A Distributed Database Based on Paxos \\
Examination:        & \bf Computer Science Tripos, July 2013        \\
Word Count:         & \bf wordcount \\
Project Originator: & Charlie Shepherd                    \\
Supervisor:         & Stephen Cross                    \\
\end{tabular}
}


\section*{Original Aims of the Project}

\subsection*{Project aims}

I aim to implement a distributed database. This will be based on the Paxos protocol.

\subsubsection*{Paxos}

The first half of the project is to implement the Paxos protocol. This will be done in a module,
providing an interface which the database can then use to

The project must correctly implement the Paxos protocol.  The library must be capable of forming a
running network, in particular dynamic leader election, as well as achieving consensus on a
key/value store across the network.

The database must implement a subset of SQL, specifically. The database must have all ACID
properties, that is.

\section*{Work Completed}

All that has been completed appears in this dissertation.

\section*{Special Difficulties}

None.

\newpage
\section*{Declaration}

I, Charlie Shepherd of Churchill College, being a candidate for Part II of the Computer Science
Tripos, hereby declare that this dissertation and the work described in it are my own work,
unaided except as may be specified below, and that the dissertation does not contain material that
has already been used to any substantial extent for a comparable purpose.

\bigskip
\leftline{Signed: }

\medskip
\leftline{Date: \today}

\cleardoublepage

\tableofcontents

\listoffigures

\newpage
\section*{Acknowledgements}

Acknowledge various people

%%%%%%%%%%%%%%%%%%%%%%%%%%%%%%%%%%%%%%%%%%%%%%%%%%%%%%%%%%%%%%%%%%%%%%%
% now for the chapters

\cleardoublepage        % just to make sure before the page numbering
                        % is changed

\setcounter{page}{1}
\pagenumbering{arabic}
\pagestyle{headings}

\chapter{Introduction}

\section{Overview}

I set out to build a distributed database with ACID properties, based on the Paxos protocol. Paxos
was first described in Leslie Lamport's paper \emph{The Part Time Parliament} \cite{lamport98}.

\section{Motivation}

% XXX: finish!

\section{Distributed Consensus Problem}

Consensus is an integral problem in distributed systems. The consensus problem is that of getting
all nodes in a distributed system to agree on a value. Formally, an algorithm that satisfies the
consensus problem satisfies three properties:

\begin{enumerate}
\item Agreement - all nodes must decide the same value.
\item Validity - the value that is decided upon must have been proposed by some node in the
	network.
\item Termination - all nodes eventually decide on a value.
\end{enumerate}

Decision is defined as follows - a process must decide on a value only once, and cannot change the
value once it has been decided.

\subsection*{Failure Modes}

Distributed networks must be able to deal with failure in a number of forms. Nodes may fail and
stop executing completely, this is known as the \emph{fail-stop} model. In practice, a more accurate
model is the \emph{fail-recover} model. This is where a node stops executing, then resumes execution
at a later time. This more accurately models real life, where asynchronous networks mean that it
may take an arbitrarily long time to receive a message, and that messages may arrive out of order
or not at all. The most general failure model is that of \emph{Byzantine failure}, where nodes may
respond incorrectly or even deliberately lie in order to mislead the protocol. In practice this is
unlikely, and can be mediated using checksums, message digests and authentication. In the
Preparation chapter I will go into detail about the types of failure that my database can handle.

\subsection*{2 Phase Commit}

\begin{figure}[h!]
\centering
\includegraphics[scale=0.5]{figs/two-pc.eps}
\caption{\label{fig:two-pc}Two Phase Commit}
\end{figure}

2 Phase Commit (2PC) is one of the simplest consensus protocol, and one of the most brittle. The
basic message flow is shown in figure~\ref{fig:two-pc}. It has two phases:

\begin{enumerate}
\item The co-ordinator (the node initiating the protocol) sends a \msg{PROPOSE} message to each
	node in a cohort of size $N$, asking them to accept the value proposed.
\item The nodes reply with a \msg{YES} or \msg{NO} reply.
\item If all nodes respond with a \msg{YES} message, the co-ordinator sends a \msg{CONFIRM} message. Otherwise, if
	any node responds with a \msg{NO} message, it sends an \msg{ABORT} message to all nodes.
\item Nodes reply with an \msg{ACK} message, and the co-ordinator marks the transaction as
	complete when all nodes have acknowledged it.
\end{enumerate}

2PC solves the consensus problem in a failure free network. However if we can have failures then
the protocol can suffer from a number of limitations. If the co-ordinator crashes before sending
any messages, we satisfy consensus trivially. However, if the co-ordinator crashes after sending
$x$ messages, where $1 \le x < N$, the protocol cannot continue - it is blocked on the
co-ordinator resuming, and if it never resumes then some members of the cohort are blocked
permanently. In fact, once a node has sent a \msg{YES} message, it is blocked until it receives a
response. This is a big disadvantage for a concurrent system. While there are extensions to
resolve the problem of a crashing co-ordinator, these do not fix the fundamental problem of using
a blocking protocol in an asynchronous network. (These extensions often involve a ``watchdog'' or
``recovery node'' This is still not a satisfactory solution as the simultaneous crash of the
co-ordinator and a cohort member means the state of the network is not recoverable (ie, we cannot
tell if the cohort member who crashed voted \msg{YES} or \msg{NO}.))

\subsection*{3 Phase Commit}

\begin{figure}[h!]
\centering
\includegraphics[scale=0.5]{figs/three-pc.eps}
\caption{\label{fig:three-pc}Three Phase Commit}
\end{figure}

3 Phase Commit (3PC) is a modification of 2PC that turns it from a synchronous protocol into an
asynchronous protocol. This is done by adding a third phase, so that we can use timeouts to assert
the state of the system at any point in time. Again the basic message flow is shown in
figure~\ref{fig:three-pc}. The three phases are:

\begin{enumerate}
\item \msg{Prepare}: the co-ordinator asks the cohort members if they can perform the operation. At
	this stage if there is a failure or timeout, the co-ordinator aborts the transaction.
\item The cohort respond with a \msg{YES} or \msg{NO}. Again, if there is a failure or timeout the cohort
	member considers the transaction aborted.
\item \msg{Pre-commit}: If the co-ordinator receives \msg{YES} messages from every member of the
	cohort, it sends a \msg{Pre-commit} message to them all, otherwise it aborts the
	transaction. It also aborts in the case of a failure or timeout.
\item \msg{Acknowledge}: If the cohort member recieves a \msg{Pre-commit} message, it replies with
	an \msg{Acknowledge} message. If the co-ordinator aborts, or there is a failure or
	timeout, the cohort member aborts.
\item \msg{Finalise}: If the co-ordinator times out, it aborts the transaction. Otherwise, when it
	has received \msg{Acknowledge} messages from every cohort member it sends a \msg{Finalise}
	message to them all.
\item \msg{Confirm}: Once a cohort member has received a \msg{Finalise} message, it commits the
	transaction, even if the co-ordinator fails. It can reply with a \msg{Confirm} message.
\end{enumerate}

This fixes the problem of node failure, but still suffers from a significant problem. If there is
a network partition, and all nodes who voted \msg{YES} are in one half and all nodes who voted
\msg{NO} are in the other, both partitions will recover into different, inconsistent states.
Brewer's CAP theorem \cite{gilbert2002} says that we cannot achieve both consensus and availability
during a network partition - we must choose one. 3PC opts for availability, but for a distributed
database with ACID properties it is preferable to have consensus. Paxos guarantees consensus at
the cost of lack of availability in a network partition.

\subsection*{Paxos}

% http://betathoughts.blogspot.co.uk/2007/06/brief-history-of-consensus-2pc-and.html
% The kernel of Paxos is that given a fixed number of processes, any majority of
% them must have at least one process in common. For example given three
% processes A, B and C the possible majorities are: AB, AC, or BC. If a decision
% is made when one majority is present eg AB, then at any time in the future when
% another majority is available at least one of the processes can remember what
% the previous majority decided. If the majority is AB then both processes will
% remember, if AC is present then A will remember and if BC is present then B
% will remember.
%
% Paxos can tolerate lost messages, delayed messages, repeated messages, and
% messages delivered out of order. It will reach consensus if there is a single
% leader for long enough that the leader can talk to a majority of processes
% twice. Any process, including leaders, can fail and restart; in fact all
% processes can fail at the same time, the algorithm is still safe. There can be
% more than one leader at a time.
%
% Paxos is an asynchronous algorithm; there are no explicit timeouts. However, it
% only reaches consensus when the system is behaving in a synchronous way, ie
% messages are delivered in a bounded period of time; otherwise it is safe. There
% is a pathological case where Paxos will not reach consensus, in accordance to
% FLP, but this scenario is relatively easy to avoid in practice.

Paxos is a generalisation of 2PC and 3PC that handles more failure modes. While 3PC can handle
failures, it only handles \emph{fail-stop} failures, not \emph{fail-recover} failures. This is
unfortunate, as computer networks are asynchronous networks, and can therefore be modelled using a
fail-recover model to account for arbitrary delays from messages being sent to being received
(fail-stop would mean messages can only be dropped, not delayed for an arbitrary length of time).

Paxos relies on the fact that any majority $M$ has at least one member in common. Therefore if we
enforce the constraint that any decision must be made by a majority, then once a decision is made,
in any subsequent decision there will be at least one member who was present in the previous
decision. This is the central tenet of Paxos, where a majority is known as a \emph{quorum}.

A key advantage of Paxos is that it can guarantee consistency during network partitions,
unlike 3PC which gives availability instead.

However, a potential drawback is that Paxos has a pathological case where it will not make
progress, ie, there is no upper bound on the time it will take to achieve consensus (it may never
terminate). This is due to the FLP impossibility proof \cite{fischer85}, which states that it is
impossible to guarantee liveness in an asynchronous network where there is at least one failure.
In practise it is a situation that is relatively easy to avoid, and is not a serious concern.

\section{Databases}

\subsection*{ACID}

Commonly all databases provide ACID properties. ACID stands for:

\begin{itemize}
\item Atomic - an operation is either performed or is not performed.
\item Consistent - the database remains in a consistent state at all times.
\item Isolated - one operation cannot see the intermediate state due to another operation occurring
	at the same time.
\item Durable - once an operation is ``committed'' it is permanently stored - the effects of it will
	not be lost.
\end{itemize}

There are many reasons why ACID is a key requirement of most databases. ACID makes it far easier
for application developers to reason about concurrency in a system where there may be multiple
clients reading and writing the same data simultaneously. It also enable effective abstraction, as
the atomicity requirement in particular gives the well defined characteristics when an operation
or transaction fails. ACID also provides clear guarantees with regard to data stability and
longevity, in particular how this relates to consistency, so that users can be clear as to the
overall state of the system.

\subsection*{Centralisation vs Distributed}

Databases are generally found in one of two topologies - centralised or distributed. These
different topologies are used in different situations and to different ends.

A centralised database is a single node in a single location.
In general, centralised databases are simpler in design and on the whole faster than distributed
databases.
Many general properties of centralised vs. distributed systems apply to centralised databases -
they are easier to perform organise, to edit, to query and to backup. May slow down under load. It
is easier to maintain the integrity of data on a centralised database, as there is only one
``current'' version of the data under consideration.

In contrast, distributed databases are more resilient and scalable than centralised databases.
There is no longer a single point of failure, and they can be extended by adding nodes, although
this will have a point of diminishing returns. Distributed databases are often more complex in
design, as they have to ensure consistency between nodes (although there is a recent trend to
relax this constraint in certain ``NoSQL'' databases. Here I will only consider traditional RDBMS
systems). In particular, distributed databases can be slow accessing non local data.

\subsection*{Existing Paxos databases}

Paxos has recently become a very popular algorithm for ensuring distributed
consensus. One good example is Google using Paxos for a distributed lock
service called ``Chubby'' \cite{chandra07}. This underpins their BigTable
distributed database which is used across Google.

Apache Zookeeper, a centralised service for providing distributed services such
as synchronisation, naming and configuration management, also uses a Paxos
based protocol called ZAB.

\cleardoublepage

\chapter{Preparation}

\section{Paxos}

\subsection{Introduction}

Paxos is a distributed consensus algorithm. It was developed by Leslie Lamport at Microsoft
research when he was trying to disprove its existence. Paxos is failure tolerant for up to $F$
simultaneous failures in a network of $2F + 1$ nodes.

Paxos is actually a family of protocols, based around the same main algorithm. The Paxos algorithm
is an algorithm for agreeing on a single value across a network of processors. Paxos has several
safety and liveness properties:

\paragraph{Safety Guarantees}

\begin{itemize}
\item Integrity - every correct process decides at most one value, and if it decides some value v,
	then v must have been proposed by some process.
\item Agreement - Every correct process must agree on the same value.
\item Non-triviality - Only a value that has been proposed can be chosen.
\end{itemize}

% Consistency
%     At most one value can be learned (i.e., two different learners cannot learn different values).[8][9]
% Only a single value is chosen
%
% Liveness(C;L)
%     If value C has been proposed, then eventually learner L will learn some value (if sufficient processors remain non-faulty).[9]
%
%
% A process never learns that a value has been chosen unless it has been

\paragraph{Liveness Properties}

\begin{itemize}
\item Some proposed value is eventually chosen
\item If a value is chosen, a process eventually learns it
\item Termination - Every correct process decides some value.
\end{itemize}

% \begin{enumerate}
% 	\item Consistency: Only one value is chosen.
% % XXX: wtf, how can we guarantee liveness, yet liveness is impossible??
% 	\item Liveness: If a value is proposed, eventually some value is chosen.
% 	\item Non-triviality: only proposed values may be chosen.
% \end{enumerate}

Paxos can tolerate certain kinds of failures. These are: messages being lost, delayed, repeated or
delivered out of order. It is correct even with multiple leaders, and will reach consensus if
there is a single leader that talks to a majority of processes twice.

% Paxos can tolerate lost messages, delayed messages, repeated messages, and
% messages delivered out of order. It will reach consensus if there is a single
% leader for long enough that the leader can talk to a majority of processes
% twice. Any process, including leaders, can fail and restart; in fact all
% processes can fail at the same time, the algorithm is still safe. There can be
% more than one leader at a time.
%
% Paxos is an asynchronous algorithm; there are no explicit timeouts. However, it
% only reaches consensus when the system is behaving in a synchronous way, ie
% messages are delivered in a bounded period of time; otherwise it is safe. There
% is a pathological case where Paxos will not reach consensus, in accordance to
% FLP, but this scenario is relatively easy to avoid in practice.

However Paxos cannot tolerate ``rogue'' processes, that is, processes deliberately sending
malicious or incorrect messages. There is a variant of Paxos called \emph{Byzantine Paxos} which
can tolerate this, albeit at a failure tolerance of $F$ for $3F + 1$ nodes. I will not go into
this variant further in this dissertation.

\subsection{Definitions}

\subsubsection*{Proposal Numbering}

One of the assumptions that Paxos makes is that every proposal has a unique proposal number. This
is necessary so that proposals have a \emph{total order}, ie we can compare any two proposals to
find the maximum ordered proposal. The conventional way to achieve this is to define a proposal
number as a 2-tuple of (sequence number, node address). These can be compared lexicographically,
and as node addresses are unique, every proposal number will be unique. In practice I plan to use
UUIDs as node identifiers, in order to be confident on uniqueness.

\subsubsection*{Quorums}

Paxos relies on quorums to ensure that even in the event of failures, consistency is preserved. A
quorum is defined as a majority of nodes. In some versions of Paxos, weighting can be used to
define quorums, however here I will use unweighted quorums.

\subsubsection*{Messages}

Paxos utilises several message types:

\begin{itemize}
\item \msg{Prepare(n)}, where $n$ is the proposal number of the message.
\item \msg{Promise(n, v)}, where $n$ is the highest accepted proposal number by the recipient and
	$v$ is the value of that proposal.
\item \msg{AcceptRequest(n, v)}, where $n$ is the proposal number of the proposal and $v$ is the
	value of that proposal.
\item \msg{AcceptNotify(n, v)}, where $n$ is the proposal number of the proposal and $v$ is the value
	of that proposal.
\end{itemize}

\subsection{How it works}

In \emph{Paxos Made Simple} \cite{lamport01}, the actions of a Paxos node are split into three
roles - Proposer, Acceptor and Learner. Explaining it in terms of these roles is simpler, however
in practise they are combined into one client.

\subsubsection*{Proposer}

\paragraph{Phase 1}

To start a round of Paxos, the Proposer sends out a \msg{Prepare(n)} message to the acceptors in
the network, with a unique proposal number, $n$, as outlined before. This proposal number $n$ must
be higher than any proposal numbers it has sent for this instance of Paxos.

\paragraph{Phase 2}

If the Proposer receives replies from a quorum of acceptors, $Q$, (that is, a majority of
acceptors) to its \msg{Prepare(n)} message, it sends the message \msg{AcceptRequest(n, v)} to all
$q \in Q$, where
% XXX: how do i say this in a formal way?? email stephen?
$v := $ the value of the maximum \msg{Promise(n, v)} received.
%$v \buildrel \text{d{}ef}\over = $

\subsubsection*{Acceptor}

Acceptors need to store a few variables. An acceptor $A$ stores:
\begin{itemize}
\item $\rho$ - the greatest proposal number that $A$ has received in a \msg{Prepare} message and
	responded to with a \msg{Promise} message.
\item $\eta$ - the highest-numbered proposal $A$ has accepted (initialised to $0$).
\item $\nu$ - the value of proposal $\eta$ (initialised to \verb+null+).
\end{itemize}

\paragraph{Phase 1}

If an acceptor $A$ receives a message \msg{Prepare(n)}, and $n > \rho$, then it replies with the
message \msg{Promise(n, $\bm{\eta}$, $\bm{\nu}$)}, meaning that it will not accept any proposals numbered less
than $n$.

\paragraph{Phase 2}

If an acceptor receives a message \msg{AcceptRequest(n, v)}, if $n > \rho$ it accepts the
proposal, settings $\eta := n$ and $\nu := v$. It also notifies Learners in the network of its
decision by sending an \msg{AcceptNotify(n, v)} message to them.

% XXX: is this PMS or PTP??
\emph{Paxos Made Simple} \cite{lamport01} describes several different ways of notifying Learners
of accepted proposals - we can specify a \emph{distinguished Learner}, who then notifies
other Learners when a quorum of Acceptors has accepted a proposal, we can specify several
distinguished Learners, or we can broadcast all \msg{AcceptRequest} messages to all Learners.
While this method generates $L\times A$ messages (where $L$ is the number of Learners in the
network and $A$ is the number of Acceptors in the network), it is the simplest and the most
resilient to failures, and therefore the one I have chosen to implement. By contrast, the first
method generates $A + L$ messages, and the second generates $D\times A + L$ messages, where $D$ is
the size of the set of distinguished Learners.

\subsubsection*{Learner}

% XXX: pseudo code?
Learners must store a map of proposal numbers $\rightarrow$ acceptor IDs. When a Learner receives a
message \msg{AcceptNotify(n, v)} from an Acceptor $A$, it must add $A$ to the set of acceptors who
have accepted proposal $n$. When this set becomes a quorum (ie, the size of the set $S$ becomes
greater than $N / 2$, where $N$ is the size of the network), the Learner can ``learn'' the $v$ as
% XXX: ooh, preposition?
the value decided on for that instance of Paxos. Because of the properties of Paxos, once a quorum
% XXX: do i need to explain this?
of acceptors has accepted $v$, the value of that instance will never change.

\subsection*{Examples}

\subsubsection*{Normal Behaviour}

\begin{figure}[h!]
\centering
\lwincludegraphics{figs/paxos-msg-flow-usual.eps}
\caption{\label{fig:paxos-usual}Paxos Message Flow: Usual Behaviour}
\end{figure}

Figure~\ref{fig:paxos-usual} (page~\pageref{fig:paxos-usual}) shows the message flow for a
complete round of Paxos if there are no failures and everything proceeds as expected. In this
case, then the behaviour is reasonably straightforward to follow. There are four message delays
until the proposed value is learnt.

\begin{enumerate}
\item \msg{Prepare(n):} The Proposer sends a \msg{Prepare} message to the Acceptors.
\item \msg{Promise(n):} The Acceptors all accept the proposal, as they have not seen any
	\msg{Prepare} requests yet (and therefore trivially cannot have promised to accept a
	proposal higher than $n$).
\item \msg{AcceptRequest(n, v):} The Proposer sends a value $v$, with the proposal number $n$, to
the Acceptors.
\item \msg{AcceptNotify(n, v):} The Acceptors have not promised to accept a proposal numbered
	greater than $n$, so they accept $v$ as the value for the round of Paxos and notify the
	Learner.
\end{enumerate}

\subsubsection*{Acceptor Failure}

\begin{figure}[h!]
\centering
\lwincludegraphics{figs/paxos-msg-flow-one-acceptor-fail.eps}
\caption{\label{fig:paxos-acceptor-fail}Paxos Message Flow: Acceptor Failure}
\end{figure}

Figure~\ref{fig:paxos-acceptor-fail} (page~\pageref{fig:paxos-acceptor-fail}) shows the same
scenario, but with a single Acceptor failing. In this case there is still a quorum of Acceptors
available, so behaviour carries on as normal. There are still four message delays until the
proposed value is learnt.

\begin{enumerate}
\item \msg{Prepare(n):} The Proposer sends a \msg{Prepare} message to the Acceptors.
\item Failure: Acceptor 3 fails.
\item \msg{Promise(n):} Acceptors 1 and 2 still accept the proposal, for the same reason as
	before.
\item \msg{AcceptRequest(n, v):} The quorum size is 2, so a quorum of acceptors has responded, and
	the Proposer can continue as normal.
\item \msg{AcceptNotify(n, v):} The Learner receives \msg{AcceptNotify} messages from a quorum of
	Acceptors, so also continues as normal, learning the value $v$.
\end{enumerate}

\subsubsection*{Partition}

\begin{figure}[h!]
\centering
\lwincludegraphics{figs/paxos-msg-flow-partition.eps}
\caption{\label{fig:paxos-partition}Paxos Message Flow: Partition}
\end{figure}

% This is to get this figure on the next full page, ie *before* the duelling section rather than
% on a random page after it...
\begin{figure}[p]
\centering
\lwincludegraphics{figs/paxos-msg-flow-duelling.eps}
\caption{\label{fig:paxos-duelling}Paxos Message Flow: Duelling Proposers}
\end{figure}

Figure~\ref{fig:paxos-partition} (page~\pageref{fig:paxos-partition}) shows what happens in a
network partition. A second Proposer is able to continue the instance of Paxos, and no
inconsistencies are allowed to occur in the network, as the left half of the partition is unable
to make progress until the partition is resolved, at which point it can learn the decisions made
by the half of the partition that was able to make progress.

\begin{enumerate}
\item \msg{Prepare(n):} The Proposer sends a \msg{Prepare} message to the Acceptors, but is only
	able to successfully message one of them.
\item \msg{Prepare(n):} A second Proposer simulataneously sends a \msg{Prepare} message to the
	Acceptors, and is able to reach a quorum.
\item \msg{Promise(n):} The Acceptor on the left side of the split replies with a \msg{Promise},
	but does not make up a quorum so...
\item the first Proposer times out. It could restart with a higher proposal number, but in this
	case it will be unable to make any progress until the partition is resolved.
\item \msg{AcceptNotify(n, v):} The Acceptors make up a quorum so are able to proceed through to
	completion.
\end{enumerate}

\subsubsection*{Duelling Proposers}

Figure~\ref{fig:paxos-duelling} (page~\pageref{fig:paxos-duelling}) shows the worst case scenario
for Paxos - duelling Proposers. This is the pathological case for Paxos where it is possible for
it to never make progress - this eventuality is unlikely because the messages from the proposers
must interleave so that there is no contiguous sequence of \msg{Prepare} and \msg{AcceptRequest}
messages, if any single Proposer manages to send a \msg{Prepare} message followed by a
subsequent \msg{AcceptRequest} message to a quorum of Acceptors, consensus will be achieved.

\begin{enumerate}
\item \msg{Prepare(n):} The Proposer sends a \msg{Prepare} message to the Acceptors, but is only
	able to successfully message one of them.
\item \msg{Prepare(n):} A second Proposer simulataneously sends a \msg{Prepare} message to the
	Acceptors, and is able to reach a quorum.
\item \msg{Promise(n):} The Acceptor on the left side of the split replies with a \msg{Promise},
	but does not make up a quorum so...
\item the first Proposer times out. It could restart with a higher proposal number, but in this
	case it will be unable to make any progress until the partition is resolved.
\item \msg{AcceptNotify(n, v):} The Acceptors make up a quorum so are able to proceed through to
	completion.
\end{enumerate}

\subsection{MultiPaxos}

MultiPaxos is multiple rounds of Paxos occuring at the same time. The way this is outlined in
\emph{Paxos Made Simple} \cite{lamport01} is by a form of leader election. I will outline this
here but, for simplicity, in my project I will simply use multiple Paxos rounds to form the basis
of a \emph{distributed operation log}. This will be explained in more detail in the Implementation
chapter.

\section{ACID}

\subsubsection*{Atomic}

Atomicity means that either an operation completes, or it does not, ie, that the database is not
left in a ``halfway'' state. This means that we do not need to worry about cleaning up after an
operation, if it does not succeed we can simply retry it, without needing to worry about the state
it has left the database in. Atomicity also applies to transactions in the same way - if we have a
transaction as an operation composed of smaller single operations (for example, INCR B, READ A
$\rightarrow$ X, WRITE X + 10 $\rightarrow$ A), we don't want some of the operations to complete
and some not to. In this example, we may have the constraint $A = 10\times B$. If B was
incremented but then the transaction failed before A was updated, we would leave the database in
an inconsistent state.

\subsubsection*{Consistent}

Consistency is very closely related to atomicity, as consistency refers to a property of the
database, and atomicity refers to a property of operations performed on the database. In the
example used for atomicity, we had a constraint on the database (that $A = 10\times B$). We want operations
(or transactions) to transform the database from one consistent state to another. This becomes
more pertinent in distributed databases, as we may receive operations in one order at one node,
and in a different order at another node. If we apply the operations in the order that we receive
them, the two nodes are likely to be in inconsistent states. XXX: define consistency in a
distributed system.

\subsubsection*{Isolated}

Isolation is the property that ensures that if one transaction is in the process of completing,
another transaction cannot see the intermediate state. It ensures that transactions that execute
concurrently to each other are also invisible to each other - a transaction cannot tell, and does
not need to worry about whether another transaction is executing or not.

\subsubsection*{Durable}

The durable property refers to the permanance of a transaction. Once a database reports that a
transaction has been commited, it should persist in the database even in the face of power loss or
crashes. This is typically achieved by using a write-ahead log that is append only, so that in the
event of a crash, the database can reconstruct the state of all transactions reported as
``committed''. Distributed systems have an added complication of co-ordinating this property
across the network before the transaction can be reported as ``committed''.

\section{Software Engineering}

\subsection{Module Dependencies}

I chose Twisted in order to make implementing the protocol easier. Twisted provides a lot of
support for implementing protocols and helper classes etc. Also as I am using a state machine
approach to implementing DBP and Paxos, twisted's asynchronous system works very well with this
approach.

\subsection{Programming Language}

There were several options for which programming language to use for this project.  C compiles to
very fast code, and gives a lot of control over the behaviour of the whole system.  However it is
very verbose. Another option was erlang. Erlang was designed for networking and exhibits a high
degree of parallelism. However it is a very different paradigm and I have never used it before.
In the end I went with Python, a language I am very familiar with, and which is very easy to
prototype and do rapid development in.

\section{Requirements Analysis}

Requirements for my system:
Transactions
ACID properties

\subsection{Software Development Process}

I considered using the waterfall model, but it didn't fit in with my requirement to do fast
prototyping (as I needed to gain a better understanding of Paxos). I settled on the spiral model
because of XXX:

\subsection{Version Control and Backup Strategy}

For version control I decided to use Git, as it is a system I am familiar with, and serves my need
both as a VCS and as a remote backup. In my experience it is more usable than other DVCSs such as
Bazaar, Darcs or Mercurial, and much faster than centralised VCSs such as CVS or SVN. Using Git I
backed up my project both to bitbucket, an online repository service which provides free private
repository hosting, and to my own server.

For backups I use the PWF to develop my project on. As it is stored in Git I can back it up simply
by pushing the repo to other hosts. I currently have it backed up to two other hosts.

\subsection{Testing}

Testing is really important
network
lots of effects
lots of subsystems

\subsubsection{Unit Testing}

unit testing really key
regression testing
testing different subsystems

\subsubsection{Integration Testing}

test program etc
see simple change propagate across complex system



\cleardoublepage
\chapter{Implementation}

\section{Paxos Design}

\subsection{Protocol Design}

I started off with Paxos. I wanted to iterate quickly from prototype to prototype, adding features
slowly as I understood the protocol more, as I found it confusing and wasn't sure how to implement
in code various concepts outlined in the academic papers I read (mainly \emph{Paxos Made Simple}
\cite{lamport01}).

My initial prototype was a synchronous model that sent messages internally. I quickly decided to
change to Twisted, as I hoped the support it would give me would make writing the protocol easier.

\subsubsection*{Messages}
I initially decided to use a class hierachy to define messages, and to use a simple
serialization/deserialization technique, transmitting messages in the form
\verb+"<message type>":<proposal serialization>+. I decided to send messages as simple strings
over the network for a number of reasons - for a prototype implementation speed was not my primary
concern, iterating towards the most complete and correct solution in a reasonable period of time
was. Furthermore, even if my priority was speed, optimising the message format felt like a
premature optimisation, and using a binary format would vastly hinder my debugging.

In hindsight restricting messages to only a combination of message type and proposal attributes
was unnecessarily restrictive. An advantage of constructing them in this way was to prevent typos
in creating messages (cf. \verb+send(Msg({"msg_type": "accept_requst", ...}))+
and \verb+send(Msg({"msg_type": "accept_request", ...}))+).

Although security was not a major concern for this project, I wanted to be able to serialize
arbirtrary objects without allowing remote code execution on a host running my software - even
though it was only running on local host this still seemed unnecessarily risky. Fortunately Python
has a builtin function called \verb+literal_eval+ which only evaluates literals (strings, tuples,
lists and dictionaries), and nothing unsafe (classes, functions) which could be used to run
arbritrary code.

I eventually moved to sending a dictionary in plain string format over the network. This allowed
me to specify arbitrary key/value pairs without having to add in extra support for them (a
limitation I initially struggled with before moving to this format). This greatly simplified a lot
of logic, at the cost of trusting that messages received are well formed.  However, this is not
too problematic for several reasons - firstly, if the message is not well formed, the code will
throw an exception, which will be caught by the message handler and discarded. Paxos allows for
any message to be ignored or dropped and still guarantees correctness (it must do this in order to
work if a packet is dropped in the network or delayed indefinitely). If correctness of the message
needed to be verified, it would be relatively easy to add a checksum of some kind, a very simple
form of this is found in the \emph{netstring} format (defined by DJB at
\verb+http://cr.yp.to/proto/netstrings.txt+). This is easily added to my classes by making them
inherit from \verb+NetstringReceiver+ rather than \verb+DatagramProtocol+, and using
the \verb+stringReceived+ callback rather than the \verb+datagramReceived+ callback. In any case,
it is beyond the scope of this project to deal with malicious nodes, so I did not spend a
significant amount of time considering this problem.

\subsubsection*{Hosts}

I first used a tuple of \verb+(IP Address, Port Num)+ to identify hosts. However there turned out
to be a number of problems with this. Firstly as an identifying scheme it is not persistent across
interfaces. Also there is a significant problem if a node needs to identify itself (a pertinent
example is for the \op{TRYLOCK} operation). After I changed the message format to a generic
dictionary format, I wanted to add an attribute specifying the sender. This is difficult to do
using tuple format, as it is non-trivial for a host to obtain its own IP address, it may be on a
local network or behind a NAT for example.

I updated it to use a GUID. This is better for several reasons - the host knows its own address
etc

There is the problem of a node lying about its identity, for example forging a \op{TRYLOCK}
request. This is a more pertinant problem because I am using UDP, which is easier to forge than
TCP. Although a fully secure implementation is beyond the scope of this project, one potential
solution would be to sign or authenticate messages, and include this signature or MAC with the
message.

\subsection{Protocol Extras}

On top of the basic Paxos protocol I added some extra features to the protocol, in particular,
node discovery and bootstrap; and heartbeat monitoring of nodes to detect them leaving the
network.

Keeping an accurate idea of network membership is a key requirement of Paxos. Each node in the
network needs to have a good idea of the number of nodes in the network in order to have an
accurate estimate of the quorum size. If a node underestimates the quorum size, the network may
become inconsistent, as a node may ``learn'' a value erroneously. On the other hand, if we
overestimate the quorum size, we may not make any progress, waiting for more responses than there
are nodes in the network. In practice, the first problem is more problematic than the second,
which is only temporary - as long as the node eventually accurately learns the quorum size the
network will make progress, however if the network develops inconsistencies these are much harder
to resolve.

% Consensus algorithms need a strong failure detector \cite{chandra96}.

\subsubsection{Bootstrap}

A node connects to the network by connecting to a \emph{bootstrap node}. In my current
architecture this can be any node, however in a different model it may be a specific node, see the
Evaluation chapter for more details involving scaling. When a node $N$ connects to the bootstrap
node $B$, it sends an \msg{EHLO} message. $B$ replies with a \msg{Notify} message containing all the
hosts $B$ is aware of. $N$ then sends each of these in turn \msg{EHLO} message, making each of them
aware of its presence, and getting a \msg{Notify} message from each othem. This is repeated until
there are no nodes it has not learned of. $N$ is then fully integrated into the network.

Note that this bootstrap method is $N^2$ in the number of messages sent. There are other ways to
do bootstrap that are more efficient in the number of messages sent (for instance a DHT or a
supernode hierachy). While this architecture is out of the scope of this project these options are
discussed later on.

\subsubsection{Heartbeat}

In order to identify when a node leaves the network, when a node initialises it starts a timer on
a configurable timeout and sends a \msg{Ping} message to every node in its \verb+hosts+ attribute.
It then copies the \verb+hosts+ set to a \verb+timeout_hosts+ set. As it receives a reply from a
host it removes that host from the \verb+timeout_hosts+ set. When the timeout fires, any nodes who
have not sent a \msg{Pong} in reply are left in the \verb+timeout_hosts+ set, and are removed from
the \verb+hosts+ set.

\subsection{MultiPaxos}

The simple way MultiPaxos is implemented is multiple instances of Paxos operating in parallel.
Paxos is implemented as a state machine, with the instance state as a Python dictionary and
transitions as methods. A simple way to implement MultiPaxos is to have every transition method
take an instance dictionary as an argument and operate on that. One problem with this approach is
that if any initial messages (\msg{Prepare}, \msg{AcceptRequest}, etc) are not received, the
initial state is not constructed.

For certain messages that are not linked to any particular instance of Paxos, the message
attribute \verb+"instance_id"+ is sent with value \verb+None+, for all other messages the instance
id is an integer referring to the OID of the instance of Paxos running. For example, all messages
corresponding to OID 2 in the operation log would have the attribute \verb+"instance_id": 2+ set
in their message dictionary.

\section{Database Design}

\subsection{Design}

\subsubsection{Basic Operations}

The database is implemented as a datastructure that can have operations performed on it. These
operations can be \emph{serialised} and \emph{deserialised}. This involves converting the objects
in memory into an \emph{on-the-wire} format that can be sent over the network and converted back
into a Python object at the receiving node.

The main problem for designing a distributed database then becomes deciding on an order for these
operations that is consistent across every node.

A basic operation is one that occupies a single slot of the operation log.

\begin{tabular}{ | c | c | p{7cm} | }
  \hline
  {\bf Operation} & {\bf Meaning} \\ \hline
  NOP & Do nothing \\ \hline
  ASSIGN(k, v) & Set \verb+k := v+ in the database. \\ \hline
  ATTEMPTLOCK & Attempt to take the TX lock. \\ \hline
  UNLOCK & Release the TX lock. \\ \hline
\end{tabular}

\subsubsection{The Operation Log}

\begin{figure}[htb]
\centering
\includegraphics[scale=0.5]{figs/op-log.eps}
\caption{\label{fig:op-log}Operation Log}
\end{figure}

Deciding on a serialisation for operations is done by the operation log. This associates an
``Operation ID'' (OID) with a particular operation. In order to decide on an OID for an operation,
the code picks the next highest instance ID it hasn't seen before and starts a new round of Paxos
for that instance, trying to assert that \verb$OID := <op>$. OIDs are integers and refer to
indexes into an ordered table - the operation log.

I tried a number of different techniques for retrying in the event that the OID was associated
with another operation. This is always safe because of Paxos. In fact, this technique can even be
used to learn all the database operations performed so far, albeit reasonably inefficiently,
simply by performing a \op{READ}.

% XXX: check retry semantic divide between paxos/db

\subsubsection{Reads}

For a distributed database there is the concept of \emph{fast reads} and \emph{slow reads}. A fast
read is a read that only examines the state of the database locally, without sending anything over
the network.  A slow read involves inserting an operation into the operation log in order to
ensure that the local copy of the database is up to date, then reading from that state.

A fast read has very low latency, as it does not need to send messages to any other nodes, but it
may return out of date data. A slow read forces us to actually examine the current state of the
database by inserting a \op{READ} operation, to ensure we have obtained all transactions issued
prior to our \op{READ}.

explain

In fact, I chose to use a \op{NOP} operation rather than a \op{READ} operation, although they
would semantically be the same (as other nodes do nothing on a \op{READ}), as I felt a \op{NOP}
better represented the operation being sent (do nothing).

explain about using nops to build knowledge of db

\subsubsection{Transactions}

\begin{figure}[htb]
\centering
\includegraphics[scale=0.5]{figs/op-log-trylock.eps}
\caption{\label{fig:op-log-trylock}Transactions in the Operation Log}
\end{figure}

My first implementation of transactions was a very na\"ive global lock. In order to start a
transaction, a node inserts the \op{TRYLOCK} operation. If, when the operation is inserted, the
number of \op{TRYLOCK}s is well-bracketed (ie, there is an equal number of lock takes and
releases), then the node was successful in taking the lock. While a node holds the lock, no other
node can perform an operation (the operations are inserted in the transaction log but ignored).

elaborate



\section{Paxos Implementation}

\subsection{State Machine Architecture}


Each of these acts like independant state machines.

Mixins
getattr

\begin{tabular}{ | c | c | p{7cm} | }
  \hline
  {\bf Message type} & {\bf Handler Class} & {\bf Message action} \\ \hline
  AcceptRequest & Acceptor & Respond with AcceptNotify (if valid) and accept Proposal. \\ \hline
  AcceptNotify & Learner & Record response and if a quorum has accepted that proposal, learn it. \\ \hline
  EHLO & Node & Respond with notify. \\ \hline
  Notify & Node & Add hosts to host list. \\ \hline
  Ping & Node & Send pong.  \\ \hline
  Pong & Node & Remove host from timeout list. \\ \hline
  Prepare & Acceptor & Respond with Promise (if valid). \\ \hline
  Promise & Proposer & Respond with AcceptRequest and deal with timeouts. \\ \hline
\end{tabular}



\section{Database Implementation}

SQL
How tools affected things
Optimisation - ``what it is''
Talk about iterations

SQL parser
- what it supports

-start up costs
  - ping time etc
  - inefficiencies
  - cf. ``supernodes'' vs DHTs to organise nodes

Talk about laptop breaking

\section{Testing}

I started off implementing unit tests for the

\section{Commandline Tools}

bin/peval
etc etc
500 words


\cleardoublepage
\chapter{Evaluation}

This is where the second most amount of marks are gained.

Talk about scaling - different types of hierachy - DHT, supernodes etc

\section{Testing}

\begin{enumerate}
	\item unit test inertia
	\item test programs, see complex effects of single change.
	\item durable - network - stable storage
\end{enumerate}



\cleardoublepage
\chapter{Conclusion}

Conclude here.




\cleardoublepage

%%%%%%%%%%%%%%%%%%%%%%%%%%%%%%%%%%%%%%%%%%%%%%%%%%%%%%%%%%%%%%%%%%%%%
% the bibliography

\addcontentsline{toc}{chapter}{Bibliography}
\bibliography{refs}
\cleardoublepage

%%%%%%%%%%%%%%%%%%%%%%%%%%%%%%%%%%%%%%%%%%%%%%%%%%%%%%%%%%%%%%%%%%%%%
% the appendices
\appendix

\chapter{Project Proposal}

\vfil

\centerline{\Large Computer Science Project Proposal}
\vspace{0.4in}
\centerline{\Large Paxos: A Distributed Consensus Protocol}
\vspace{0.4in}
\centerline{\large Charlie Shepherd, Churchill College}
\vspace{0.3in}
\centerline{\large Originator: Charlie Shepherd}
\vspace{0.3in}
\centerline{\large 5$^{th}$ October 2012}

\vfil


\noindent
{\bf Project Supervisor:} Stephen Cross
\vspace{0.2in}

\noindent
{\bf Director of Studies:} Dr John Fawcett
\vspace{0.2in}
\noindent

\noindent
{\bf Project Overseers:} Dr~A.~Madhavapeddy \& Dr~M.~Kuhn


% Main document

\section*{Introduction, The Problem To Be Addressed}

Paxos is a protocol for achieving distributed consensus, developed by Leslie Lamport at
Microsoft Research.
The motivation for Paxos as a protocol is that it is capable of handling failures that other
consensus protocols cannot. Two Phase Commit (2PC) and Three Phase Commit (3PC) are two common
protocols that can be used to ensure atomic commits in a distributed system.

2PC works by having a co-ordinator node contact every node and send a proposal message. Each node
must then either respond with a commit or abort message. However, 2PC suffers from a problem, in
that if the co-ordinator node crashes after any proposal messages have been sent out, the protocol
will block. This can be resolved by establishing a timeout and, on timeout, establishing a
"recovery node" that recovers the state of the proposal and completes transaction (either by
committing or aborting). The main limitation of this protocol is that, even with this fix in
place, a fundamental limitation of the protocol is that if, while the network is in "recovery
mode", another node crashes, the entire network is blocked, and the state is unrecoverable. This
is a fairly fundamental flaw in a protocol if it is to be used over the Internet, where computer
crashes and network failures (which can be represented as a crash) are common.

3PC is an extension to 2PC which endeavours to fix this limitation, at the expense of greater
latency, by adding a third roundtrip to confirm the commit to all nodes. This means that the
protocol is asynchronous, and that node failures cannot block the protocol or cause it to fail.
However, it still has its own limitations, in particular, in the event of a network partition. If
the network is partitioned so that in one partition all nodes vote "commit" and in the other all
nodes vote "abort", both partitions will initiate recovery, and when the network merges again the
system will be in an inconsistent state. This is the limitation that Paxos was intended to solve.

\section*{Starting Point}

{\em Describe existing state of the art, previous work in this area, libraries and databases to be used.
Describe the state of any existing codebase that is to be built on.  }

My starting point will be to study the Paxos protocol. From there I will design a library
implementation of Paxos and then implement it, along with unit tests. The challenge will be
developing a complex distributed protocol, as this is something I have minimal experience of.

\section*{Resources Required}

{\em A note of the resources required and confirmation of access.}

I will mainly do my project on a virtual machine which is running on my own personal laptop.
The source code will be committed to a Git repository, which will be pushed to Bitbucket and my
own personal host. The virtual machine contents will be backed up on an external HDD for quick
recovery, although the git repository will be adequate for restoring my project if the system I am
developing it on fails.
I require no other special resources.

\section*{Work to be done}

{\em Describe the technical work.}

The project breaks down into the following sub-projects:

\begin{enumerate}

\item A study of distributed algorithms and the Paxos protocol

\item Implementing the Paxos protocol

\item Analysing several different performance metrics of the protocol implementation

\item Evaluating potential improvements to the protocol

\end{enumerate}

\section*{Success Criterion for the Main Result}

In order for the project to be a success, the project must correctly implement the Paxos protocol.
The main deliverable is a python script and library implementing Paxos. The library must be
capable of forming a running network, in particular dynamic leader election, as well as achieving
consensus on a key/value store across the network. A node must be capable of joining the network
given only the address and port of all other nodes in the network.


\section*{Possible Extensions}

{\em Potential further envisaged evaluation metrics or extensions.}

There are several potential evaluation metrics for the project.

One major metric is latency - the time for one key/value pair to be committed to the system. This
can be evaluated in a number of difference circumstances, including simulated node failure, leader
failure and network partition, and the results analysed to see how the system handles performance
under failure compared to normal conditions.

Another key metric is throughput - the maximum number of key/value pairs to be committed to the
network over a specified period of time. Again there are a number of different situations
throughput can be measured in, including load from one source, load from multiple sources and
load under failure.

A potential extension topic is dynamic node bootstrap - given the address/port pair of another
node, can a node successfully bootstrap itself into the network? Another potential topic is that
of a FUSE filesystem - as a filesystem is essentially added semantics on top of a key/value store,
it would be possible to build a distributed filesystem on top of a Paxos network, using FUSE.

\section*{Timetable: Workplan and Milestones to be achieved.}

{\em Perhaps list ten or so two-week work-packages.}

Planned starting date is 19/10/2011.

{\bf Michaelmas Term}
\begin{enumerate}

\item {\bf 19/10/2012-01/11/2012} Research distributed algorithms and the Paxos protocol, in order
to be ready to start designing the system.

Milestone: A write up of the Paxos algorithm.

\item {\bf 02/11/2012-15/11/2012} Design the protocol implementation and library layout, in order
to be reading to start the implementation.

Milestone: A design document of the implementation.

\item {\bf 16/11/2012-29/11/2012} Begin implementation of protocol library

Milestone: Basic Paxos implementation

\noindent {\bf Christmas Vacation}

\item {\bf 30/11/2012-13/12/2012} Finish implementation of library

Milestone: Paxos implementation that can coordinate distributed leader election and achieve
consensus, including unit tests.

\item {\bf 14/12/2012-27/12/2012} Slack time/Revision/Holiday break

\item {\bf 28/12/2012-10/01/2013} Prepare for progress report, start latency analysis.

Milestone: Draft progress report, latency analysis data.

{\bf Lent Term}
\item {\bf 11/01/2013-24/01/2013} Write progress report, perform throughput analysis.

Milestone: Finished Progress report, throughput analysis data.
Deadlines: Progress report deadline

\item {\bf 25/01/2013-07/02/2013} Investigate improvements to the protocol/implementation and their effect on
performance metrics.

Milestone: Improvements to protocol/implementation and revised performance data.

\item {\bf 08/02/2013-21/02/2013} Start dissertation

Milestone: Draft Introduction and Preparation sections complete

\item {\bf 22/02/2013-07/03/2013} Finish writing up dissertation

Milestone: Draft Implementation, Evaluation and Conclusion sections complete

\item {\bf 08/03/2013-21/03/2013} Proof reading and then an early submission so as to concentrate on examination revision

Milestone: Finished dissertation

Easter Vacation
\item {\bf 22/03/2013-04/04/2013} Slack time/Revision/Holiday break

\item {\bf 05/04/2013-18/04/2013} Slack time/Revision/Holiday break

Easter Term
\item {\bf 19/04/2013-02/05/2013} Slack time/Revision/Holiday break

\item {\bf 03/05/2013-16/05/2013} Slack time/Revision/Holiday break

Deadlines: Official dissertation submission deadline

\end{enumerate}


\end{document}

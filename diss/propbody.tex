\vfil

\centerline{\Large Computer Science Project Proposal}
\vspace{0.4in}
\centerline{\Large PDB: A Distributed Database Based on Paxos}
\vspace{0.4in}
\centerline{\large Charlie Shepherd, Churchill College}
\vspace{0.3in}
\centerline{\large Originator: Charlie Shepherd}
\vspace{0.3in}
\centerline{\large 18$^{th}$ October 2012}

\vfil


\noindent
{\bf Project Supervisor:} Stephen Cross
\vspace{0.2in}

\noindent
{\bf Director of Studies:} Dr John Fawcett
\vspace{0.2in}
\noindent

\noindent
{\bf Project Overseers:} Dr~A.~Madhavapeddy \& Dr~M.~Kuhn


% Main document

\section*{Introduction, The Problem To Be Addressed}

{\em Paxos} is a protocol for achieving distributed consensus, developed by Leslie Lamport in 1991.

The motivation for Paxos as a protocol is that it is capable of handling failures that other
consensus protocols cannot. {\em Two Phase Commit} (2PC) and {\em Three Phase Commit} (3PC) are two common
protocols that can be used to ensure atomic commits in a distributed system.

2PC works by having a co-ordinator node contact every node and send a proposal message. Each node
must then either respond with a commit or abort message. However, 2PC suffers from several
problems, mainly that it is a blocking protocol. This means that if the co-ordinator fails, and
then a node subsequently fails, the network will deadlock, as 2PC is not able to recover from that
failure situation.

3PC is an extension to 2PC which endeavours to fix this limitation, at the expense of greater
latency, by adding a third roundtrip to confirm the commit to all nodes. This means that the
protocol is asynchronous, and that node failures cannot block the protocol or cause it to fail.
However, it still has its own limitations, in particular, in the event of a network partition. If
the network is partitioned so that in one partition all nodes vote ``commit'' and in the other all
nodes vote ``abort'' both partitions will initiate recovery, and when the network merges again the
system will be in an inconsistent state. This is the limitation that Paxos was intended to solve.

My project will be to design and implement a distributed database, built on the Paxos protocol.


\section*{Starting Point}

My starting point will be to study the Paxos protocol, as well as research distributed databases.
From there I will develop a library implementing Paxos, along with unit tests. I will then design
and implement a distributed database on the Paxos library. The challenge will be to implement a
complex distributed protocol and then utilise it for a database, as these are both areas I have
little experience of.

\section*{Resources Required}

I will mainly do my project on a virtual machine which is running on my own personal laptop.
The source code will be committed to a Git repository, which will be pushed to Bitbucket and my
own personal host. The virtual machine contents will be backed up on an external HDD for quick
recovery, although the git repository will be adequate for restoring my project if the system I am
developing it on fails.
I require no other special resources.

\section*{Work to be done}

The project breaks down into the following sub-projects:

\begin{enumerate}

\item A study of distributed algorithms and the Paxos protocol

\item A study of distributed databases

\item Implementing the Paxos protocol

\item Designing the distributed database

\item Implementing the distributed database

\item Evaluating the performance of the database

\end{enumerate}

\section*{Success Criteria for the Main Result}

In order for the project to be a success, the following must be true:

\begin{enumerate}

\item The project must correctly implement the Paxos protocol.
	The library must be capable of forming a running network,
	in particular dynamic leader election,
	as well as achieving consensus on a key/value store across the network.

\item The database must implement a subset of SQL, specifically:
\begin{enumerate}
\item A single table with a static name
\item SELECT/INSERT
\item WHERE
\item GROUP BY
\item ORDER BY
\item Aggregation
\end{enumerate}

\item The database must have all ACID properties, that is:
\begin{enumerate}
\item Atomic
\item Consistent
\item Isolated
\item Durable
\end{enumerate}

\end{enumerate}

\section*{Evaluation Topics}

There are several potential evaluation metrics for the project.

One major metric is transaction latency - the time for a transaction to be committed to the system. This
can be evaluated in a number of difference circumstances, including simulated node failure, leader
failure and network partition, and the results analysed to see how the system handles performance
under failure compared to normal conditions.

Another key metric is transaction throughput - the maximum number of transactions committed to the
network over a specified period of time. Again there are a number of different situations
throughput can be measured in, including load from one source, load from multiple sources and
load under failure.

I will also investigate the advantages of a distributed database over a normal single-server
database, particularly in terms of scalability. I will also consider how performance is affected
by the ratio of writing clients to reading clients.

\section*{Possible Extensions}

A clear possible extension is to investigate various different modifications to Paxos in order
to try to optimise the database for certain performance characteristics
(e.g. fast reads, but slow writes),
and to assess the usefulness and efficiency of these modifications.

Another possible extension is to investigate how the database performs when various ACID
properties are relaxed, and to measure and analyse how the performance gains compare with the
tradeoffs made.


\section*{Timetable: Workplan and Milestones to be achieved.}

\setlength\parindent{0pt}
\parskip = \baselineskip

Planned starting date is 19/10/2011.

\subsection*{Michaelmas Term}

{\bf 19/10/2012-01/11/2012} Research distributed algorithms and the Paxos protocol; design the
protocol implementation and library layout.

Milestone: A write up of the Paxos algorithm and a design document of the implementation.

{\bf 02/11/2012-15/11/2012} Begin the protocol implementation.

Milestone: Basic Paxos implementation.

{\bf 16/11/2012-29/11/2012} Finish implementation of Paxos library.

Milestone: Paxos implementation that can coordinate distributed leader election and achieve
consensus, including unit tests.

\subsection*{Christmas Vacation}

{\bf 30/11/2012-13/12/2012} Research distributed databases and design the database implementation.

Milestone: A write up of research on distributed databases and a design document for the database.

{\bf 14/12/2012-27/12/2012} Slack time/Revision/Holiday break.

{\bf 28/12/2012-10/01/2013} Prepare for progress report, start database implementation.

Milestone: Draft progress report, initial database implementation.

\subsection*{Lent Term}
{\bf 11/01/2013-24/01/2013} Write progress report, finish database implementation.

Milestone: Finished Progress report, fully functional database implementation.
Deadlines: Progress report deadline - 01/02/2013.

{\bf 25/01/2013-07/02/2013} Perform initial performance analysis on transaction, including
transaction latency and transaction throughput.

Milestone: Initial analysis data.

{\bf 08/02/2013-21/02/2013} Perform detailed performance analysis comparing distributed and
centralised servers, and on failing and partitioned networks.

Milestone: Analysis data on server models and on performance during failure.

{\bf 22/02/2013-07/03/2013} Investigate improvements to the protocol/implementation and their effect on
performance metrics.

Milestone: Improvements to protocol/implementation and revised performance data.

{\bf 08/03/2013-21/03/2013} Start dissertation.

Milestone: Draft Introduction and Preparation sections complete.

\subsection*{Easter Vacation}

{\bf 22/03/2013-04/04/2013} Finish writing up dissertation.

Milestone: Draft Implementation, Evaluation and Conclusion sections complete.

{\bf 05/04/2013-18/04/2013} Proof reading and then an early submission so as to concentrate on
examination revision.

Milestone: Finished dissertation.

\subsection*{Easter Term}
{\bf 19/04/2013-02/05/2013} Slack time/Revision/Holiday break.

{\bf 03/05/2013-16/05/2013} Slack time/Revision/Holiday break.

Deadlines: Official dissertation submission deadline - 17/05/2013.

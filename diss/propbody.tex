\vfil

\centerline{\Large Computer Science Project Proposal}
\vspace{0.4in}
\centerline{\Large Paxos: A Distributed Consensus Protocol}
\vspace{0.4in}
\centerline{\large Charlie Shepherd, Churchill College}
\vspace{0.3in}
\centerline{\large Originator: Charlie Shepherd}
\vspace{0.3in}
\centerline{\large 5$^{th}$ October 2012}

\vfil


\noindent
{\bf Project Supervisor:} Stephen Cross
\vspace{0.2in}

\noindent
{\bf Director of Studies:} Dr John Fawcett
\vspace{0.2in}
\noindent

\noindent
{\bf Project Overseers:} Dr~A.~Madhavapeddy \& Dr~M.~Kuhn


% Main document

\section*{Introduction, The Problem To Be Addressed}

Paxos is a protocol for achieving distributed consensus, developed by Leslie Lamport at
Microsoft Research.
The motivation for Paxos as a protocol is that it is capable of handling failures that other
consensus protocols cannot. Two Phase Commit (2PC) and Three Phase Commit (3PC) are two common
protocols that can be used to ensure atomic commits in a distributed system.

2PC works by having a co-ordinator node contact every node and send a proposal message. Each node
must then either respond with a commit or abort message. However, 2PC suffers from a problem, in
that if the co-ordinator node crashes after any proposal messages have been sent out, the protocol
will block. This can be resolved by establishing a timeout and, on timeout, establishing a
"recovery node" that recovers the state of the proposal and completes transaction (either by
committing or aborting). The main limitation of this protocol is that, even with this fix in
place, a fundamental limitation of the protocol is that if, while the network is in "recovery
mode", another node crashes, the entire network is blocked, and the state is unrecoverable. This
is a fairly fundamental flaw in a protocol if it is to be used over the Internet, where computer
crashes and network failures (which can be represented as a crash) are common.

3PC is an extension to 2PC which endeavours to fix this limitation, at the expense of greater
latency, by adding a third roundtrip to confirm the commit to all nodes. This means that the
protocol is asynchronous, and that node failures cannot block the protocol or cause it to fail.
However, it still has its own limitations, in particular, in the event of a network partition. If
the network is partitioned so that in one partition all nodes vote "commit" and in the other all
nodes vote "abort", both partitions will initiate recovery, and when the network merges again the
system will be in an inconsistent state. This is the limitation that Paxos was intended to solve.

\section*{Starting Point}

{\em Describe existing state of the art, previous work in this area, libraries and databases to be used.
Describe the state of any existing codebase that is to be built on.  }

My starting point will be to study the Paxos protocol. From there I will design a library
implementation of Paxos and then implement it, along with unit tests. The challenge will be
developing a complex distributed protocol, as this is something I have minimal experience of.

\section*{Resources Required}

{\em A note of the resources required and confirmation of access.}

I will mainly do my project on a virtual machine which is running on my own personal laptop.
The source code will be committed to a Git repository, which will be pushed to Bitbucket and my
own personal host. The virtual machine contents will be backed up on an external HDD for quick
recovery, although the git repository will be adequate for restoring my project if the system I am
developing it on fails.
I require no other special resources.

\section*{Work to be done}

{\em Describe the technical work.}

The project breaks down into the following sub-projects:

\begin{enumerate}

\item A study of distributed algorithms and the Paxos protocol

\item Implementing the Paxos protocol

\item Analysing several different performance metrics of the protocol implementation

\item Evaluating potential improvements to the protocol

\end{enumerate}

\section*{Success Criterion for the Main Result}

In order for the project to be a success, the project must correctly implement the Paxos protocol.
The main deliverable is a python script and library implementing Paxos. The library must be
capable of forming a running network, in particular dynamic leader election, as well as achieving
consensus on a key/value store across the network. A node must be capable of joining the network
given only the address and port of all other nodes in the network.


\section*{Possible Extensions}

{\em Potential further envisaged evaluation metrics or extensions.}

There are several potential evaluation metrics for the project.

One major metric is latency - the time for one key/value pair to be committed to the system. This
can be evaluated in a number of difference circumstances, including simulated node failure, leader
failure and network partition, and the results analysed to see how the system handles performance
under failure compared to normal conditions.

Another key metric is throughput - the maximum number of key/value pairs to be committed to the
network over a specified period of time. Again there are a number of different situations
throughput can be measured in, including load from one source, load from multiple sources and
load under failure.

A potential extension topic is dynamic node bootstrap - given the address/port pair of another
node, can a node successfully bootstrap itself into the network? Another potential topic is that
of a FUSE filesystem - as a filesystem is essentially added semantics on top of a key/value store,
it would be possible to build a distributed filesystem on top of a Paxos network, using FUSE.

\section*{Timetable: Workplan and Milestones to be achieved.}

{\em Perhaps list ten or so two-week work-packages.}

Planned starting date is 19/10/2011.

{\bf Michaelmas Term}
\begin{enumerate}

\item {\bf 19/10/2012-01/11/2012} Research distributed algorithms and the Paxos protocol, in order
to be ready to start designing the system.

Milestone: A write up of the Paxos algorithm.

\item {\bf 02/11/2012-15/11/2012} Design the protocol implementation and library layout, in order
to be reading to start the implementation.

Milestone: A design document of the implementation.

\item {\bf 16/11/2012-29/11/2012} Begin implementation of protocol library

Milestone: Basic Paxos implementation

\noindent {\bf Christmas Vacation}

\item {\bf 30/11/2012-13/12/2012} Finish implementation of library

Milestone: Paxos implementation that can coordinate distributed leader election and achieve
consensus, including unit tests.

\item {\bf 14/12/2012-27/12/2012} Slack time/Revision/Holiday break

\item {\bf 28/12/2012-10/01/2013} Prepare for progress report, start latency analysis.

Milestone: Draft progress report, latency analysis data.

{\bf Lent Term}
\item {\bf 11/01/2013-24/01/2013} Write progress report, perform throughput analysis.

Milestone: Finished Progress report, throughput analysis data.
Deadlines: Progress report deadline

\item {\bf 25/01/2013-07/02/2013} Investigate improvements to the protocol/implementation and their effect on
performance metrics.

Milestone: Improvements to protocol/implementation and revised performance data.

\item {\bf 08/02/2013-21/02/2013} Start dissertation

Milestone: Draft Introduction and Preparation sections complete

\item {\bf 22/02/2013-07/03/2013} Finish writing up dissertation

Milestone: Draft Implementation, Evaluation and Conclusion sections complete

\item {\bf 08/03/2013-21/03/2013} Proof reading and then an early submission so as to concentrate on examination revision

Milestone: Finished dissertation

Easter Vacation
\item {\bf 22/03/2013-04/04/2013} Slack time/Revision/Holiday break

\item {\bf 05/04/2013-18/04/2013} Slack time/Revision/Holiday break

Easter Term
\item {\bf 19/04/2013-02/05/2013} Slack time/Revision/Holiday break

\item {\bf 03/05/2013-16/05/2013} Slack time/Revision/Holiday break

Deadlines: Official dissertation submission deadline

\end{enumerate}

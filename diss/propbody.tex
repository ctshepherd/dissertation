\vfil

\centerline{\Large Computer Science Project Proposal}
\vspace{0.4in}
\centerline{\Large Paxos: A Distributed Consensus Protocol}
\vspace{0.4in}
\centerline{\large Charlie Shepherd, Churchill College}
\vspace{0.3in}
\centerline{\large Originator: Charlie Shepherd}
\vspace{0.3in}
\centerline{\large 5$^{th}$ October 2012}

\vfil


\noindent
{\bf Project Supervisor:} Stephen Cross
\vspace{0.2in}

\noindent
{\bf Director of Studies:} Dr John Fawcett
\vspace{0.2in}
\noindent

\noindent
{\bf Project Overseers:} Dr~A.~Madhavapeddy \& Dr~M.~Kuhn


% Main document

\section*{Introduction, The Problem To Be Addressed}

Paxos is a protocol for achieving distributed consensus. 

\section*{Starting Point}

{\em Describe existing state of the art, previous work in this area, libraries and databases to be used.
Describe the state of any existing codebase that is to be built on.  }

My starting point will be to study the Paxos protocol

\section*{Resources Required}

{\em A note of the resources required and confirmation of access.}

I will mainly do my project on a virtual machine which is running on my own personal computer.
Although the source code will
For this project I shall mainly use my own quad-core computer that runs Fedora Linux. Backup
will be to github and/or to an SVN repository on an external hard disk that is dumped to writable CD/DVD media.
I have another similar computer to hand should my main machine suddenly fail.
I require no other special resources.

\section*{Work to be done}

{\em Describe the technical work.}

The project breaks down into the following sub-projects:

\begin{enumerate}

\item A study of network filesystems etc

\item A study of FUSE? (Really?)

\item Developing a prototype fuse plugin in python that implements my protocol

\item Analysing several different performance metrics of the protocol

\item Looking at way to optimise the plugin and perhaps the protocol? To make it more efficient
(solely in terms of performance)

\end{enumerate}

\section*{Success Criterion for the Main Result}


In order for the project to be a success, the project must be:
\begin{enumerate}

\item Preserving consistency etc

\end{enumerate}




\section*{Possible Extensions}

{\em Potential further envisaged evaluation metrics or extensions.}



\section*{Timetable: Workplan and Milestones to be achieved.}


{\em Perhaps list ten or so two-week work-packages.}

Planned starting date is 16/10/2011.

\begin{enumerate}

\item {\bf Michaelmas weeks 2-3} Write initial Paxos functions, functions for debugging etc,
achieve consistency on a decision

\item {\bf Michaelmas weeks 4-5} Build commandline shim, transfer files between nodes, node
discovery etc

\item {\bf Michaelmas weeks 6-7} Enhance protocol support enough for directories and link FUSE
operations class

\item {\bf Michaelmas weeks 8} Enhance protocol support enough for directories and link FUSE
operations class

\item {\bf Michaelmas vacation} Finish coding: test debug etc etc

\item {\bf Lent weeks 0-1} Write progress report. Start analysis of individual nodes/local nodes,
protocol performance/limitations etc

\item {\bf Lent weeks 2-3} Run main experiments, including large scale network across CL, analyse
performance, create graphs in Matlab etc etc

\item {\bf Lent weeks 4-5} Protocol improvement

\item {\bf Lent weeks 6-7} C extension.

\item {\bf Lent weeks 8} Finish evaluation, overarching themes

\item {\bf Easter vacation:} Analyse performance increases etc

\item {\bf Easter term 0-2:}  Further evaluation and complete dissertation.

\item {\bf Easter term 3:} Proof reading and then an early submission so as to concentrate on examination revision.

\end{enumerate}

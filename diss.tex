% The master copy of this demo dissertation is held on my filespace
% on the cl file serve (/homes/mr/teaching/demodissert/)

% Last updated by MR on 2 August 2001

\documentclass[12pt,twoside,notitlepage]{report}

\usepackage{a4}
\usepackage{verbatim}

\input{epsf}                            % to allow postscript inclusions
% On thor and CUS read top of file:
%     /opt/TeX/lib/texmf/tex/dvips/epsf.sty
% On CL machines read:
%     /usr/lib/tex/macros/dvips/epsf.tex



\raggedbottom                           % try to avoid widows and orphans
\sloppy
\clubpenalty1000%
\widowpenalty1000%

\addtolength{\oddsidemargin}{6mm}       % adjust margins
\addtolength{\evensidemargin}{-8mm}

\renewcommand{\baselinestretch}{1.1}    % adjust line spacing to make
                                        % more readable

\begin{document}

\bibliographystyle{plain}


%%%%%%%%%%%%%%%%%%%%%%%%%%%%%%%%%%%%%%%%%%%%%%%%%%%%%%%%%%%%%%%%%%%%%%%%
% Title


\pagestyle{empty}

\hfill{\LARGE \bf Martin Richards}

\vspace*{60mm}
\begin{center}
\Huge
{\bf How to write a dissertation in \LaTeX} \\
\vspace*{5mm}
Diploma in Computer Science \\
\vspace*{5mm}
St John's College \\
\vspace*{5mm}
\today  % today's date
\end{center}

\cleardoublepage

%%%%%%%%%%%%%%%%%%%%%%%%%%%%%%%%%%%%%%%%%%%%%%%%%%%%%%%%%%%%%%%%%%%%%%%%%%%%%%
% Proforma, table of contents and list of figures

\setcounter{page}{1}
\pagenumbering{roman}
\pagestyle{plain}

\chapter*{Proforma}

{\large
\begin{tabular}{ll}
Name:               & \bf Martin Richards                       \\
College:            & \bf St John's College                     \\
Project Title:      & \bf How to write a dissertation in \LaTeX \\
Examination:        & \bf Diploma in Computer Science, July 2001        \\
Word Count:         & \bf 1587\footnotemark[1]
(well less than the 12000 limit) \\
Project Originator: & Dr M.~Richards                    \\
Supervisor:         & Dr M.~Richards                    \\ 
\end{tabular}
}
\footnotetext[1]{This word count was computed
by {\tt detex diss.tex | tr -cd '0-9A-Za-z $\tt\backslash$n' | wc -w}
}
\stepcounter{footnote}


\section*{Original Aims of the Project}

To write a demonstration dissertation\footnote{A normal footnote without the
complication of being in a table.} using \LaTeX\ to save
student's time when writing their own dissertations. The dissertation
should illustrate how to use the more common \LaTeX\ constructs. It
should include pictures and diagrams to show how these can be
incorporated into the dissertation.  It should contain the entire
\LaTeX\ source of the dissertation and the Makefile.  It should
explain how to construct an MSDOS disk of the dissertation in
Postscript format that can be used by the book shop for printing, and,
finally, it should have the prescribed layout and format of a diploma
dissertation.


\section*{Work Completed}

All that has been completed appears in this dissertation.

\section*{Special Difficulties}

Learning how to incorporate encapulated postscript into a \LaTeX\
document on both CUS and Thor.
 
\newpage
\section*{Declaration}

I, [Name] of [College], being a candidate for Part II of the Computer
Science Tripos [or the Diploma in Computer Science], hereby declare
that this dissertation and the work described in it are my own work,
unaided except as may be specified below, and that the dissertation
does not contain material that has already been used to any substantial
extent for a comparable purpose.

\bigskip
\leftline{Signed [signature]}

\medskip
\leftline{Date [date]}

\cleardoublepage

\tableofcontents

\listoffigures

\newpage
\section*{Acknowledgements}

This document owes much to an earlier version written by Simon Moore
\cite{moore95}.  His help, encouragement and advice was greatly 
appreciated.

%%%%%%%%%%%%%%%%%%%%%%%%%%%%%%%%%%%%%%%%%%%%%%%%%%%%%%%%%%%%%%%%%%%%%%%
% now for the chapters

\cleardoublepage        % just to make sure before the page numbering
                        % is changed

\setcounter{page}{1}
\pagenumbering{arabic}
\pagestyle{headings}

\chapter{Introduction}

\section{Overview of the files}

This document consists of the following files:

\begin{itemize}
\item {\tt Makefile} --- The Makefile for the dissertation and Project Proposal
\item {\tt diss.tex} --- The dissertation
\item {\tt propbody.tex} --- Appendix~C  -- the project proposal 
\item {\tt proposal.tex}  --- A \LaTeX\ main file for the proposal 
\item{\tt figs} -- A directory containing diagrams and pictures
\item{\tt refs.bib} --- The bibliography database
\end{itemize}

\section{Building the document}

This document was produced using \LaTeXe which is based upon
\LaTeX\cite{Lamport86}.  To build the document you first need to
generate {\tt diss.aux} which, amongst other things, contains the
references used.  This if done by executing the command:

{\tt latex diss}

\noindent
Then the bibliography can be generated from {\tt refs.bib} using:

{\tt bibtex diss}

\noindent
Finally, to ensure all the page numbering is correct run {\tt latex}
on {\tt diss.tex} until the {\tt .aux} files do not change.  This
usually takes 2 more runs.

\subsection{The makefile}

To simplify the calls to {\tt latex} and {\tt bibtex}, 
a makefile has been provided, see Appendix~\ref{makefile}. 
It provides the following facilities:

\begin{itemize}

\item{\tt make} \\
 Display help information.

\item{\tt make prop} \\
 Run {\tt latex proposal; xdvi proposal.dvi}.

\item{\tt make diss.ps} \\
 Make the file {\tt diss.ps}.

\item{\tt make gv} \\
 View the dissertation using ghostview after performing 
{\tt make diss.ps}, if necessary.

\item{\tt make gs} \\
 View the dissertation using ghostscript after performing 
{\tt make diss.ps}, if necessary.

\item{\tt make count} \\
Display an estimate of the word count.

\item{\tt make all} \\
Construct {\tt proposal.dvi} and {\tt diss.ps}.

\item{\tt make pub} \\ Make a {\tt .tar} version of the {\tt demodiss}
directory and place it in my {\tt public\_html} directory.

\item{\tt make clean} \\ Delete all files except the source files of
the dissertation. All these deleted files can be reconstructed by
typing {\tt make all}.

\item{\tt make pr} \\
Print the dissertation on your default printer.

\end{itemize}


\section{Counting words}

An approximate word count of the body of the dissertation may be
obtained using:

{\tt wc diss.tex}

\noindent
Alternatively, try something like:

\verb/detex diss.tex | tr -cd '0-9A-Z a-z\n' | wc -w/




\cleardoublepage



\chapter{Preparation}

This chapter is empty!


\cleardoublepage
\chapter{Implementation}

\section{Verbatim text}

Verbatim text can be included using \verb|\begin{verbatim}| and
\verb|\end{verbatim}|. I normally use a slightly smaller font and
often squeeze the lines a little closer together, as in:

{\renewcommand{\baselinestretch}{0.8}\small\begin{verbatim}
GET "libhdr"
 
GLOBAL { count:200; all  }
 
LET try(ld, row, rd) BE TEST row=all
                        THEN count := count + 1
                        ELSE { LET poss = all & ~(ld | row | rd)
                               UNTIL poss=0 DO
                               { LET p = poss & -poss
                                 poss := poss - p
                                 try(ld+p << 1, row+p, rd+p >> 1)
                               }
                             }
LET start() = VALOF
{ all := 1
  FOR i = 1 TO 12 DO
  { count := 0
    try(0, 0, 0)
    writef("Number of solutions to %i2-queens is %i5*n", i, count)
    all := 2*all + 1
  }
  RESULTIS 0
}
\end{verbatim}
}

\section{Tables}

\begin{samepage}
Here is a simple example\footnote{A footnote} of a table.

\begin{center}
\begin{tabular}{l|c|r}
Left      & Centred & Right \\
Justified &         & Justified \\[3mm]
%\hline\\%[-2mm]
First     & A       & XXX \\
Second    & AA      & XX  \\
Last      & AAA     & X   \\
\end{tabular}
\end{center}

\noindent
There is another example table in the proforma.
\end{samepage}

\section{Simple diagrams}

Simple diagrams can be written directly in \LaTeX.  For example, see
figure~\ref{latexpic1} on page~\pageref{latexpic1} and see
figure~\ref{latexpic2} on page~\pageref{latexpic2}.

\begin{figure}
\setlength{\unitlength}{1mm}
\begin{center}
\begin{picture}(125,100)
\put(0,80){\framebox(50,10){AAA}}
\put(0,60){\framebox(50,10){BBB}}
\put(0,40){\framebox(50,10){CCC}}
\put(0,20){\framebox(50,10){DDD}}
\put(0,00){\framebox(50,10){EEE}}

\put(75,80){\framebox(50,10){XXX}}
\put(75,60){\framebox(50,10){YYY}}
\put(75,40){\framebox(50,10){ZZZ}}

\put(25,80){\vector(0,-1){10}}
\put(25,60){\vector(0,-1){10}}
\put(25,50){\vector(0,1){10}}
\put(25,40){\vector(0,-1){10}}
\put(25,20){\vector(0,-1){10}}

\put(100,80){\vector(0,-1){10}}
\put(100,70){\vector(0,1){10}}
\put(100,60){\vector(0,-1){10}}
\put(100,50){\vector(0,1){10}}

\put(50,65){\vector(1,0){25}}
\put(75,65){\vector(-1,0){25}}
\end{picture}
\end{center}
\caption{\label{latexpic1}A picture composed of boxes and vectors.}
\end{figure}

\begin{figure}
\setlength{\unitlength}{1mm}
\begin{center}

\begin{picture}(100,70)
\put(47,65){\circle{10}}
\put(45,64){abc}

\put(37,45){\circle{10}}
\put(37,51){\line(1,1){7}}
\put(35,44){def}

\put(57,25){\circle{10}}
\put(57,31){\line(-1,3){9}}
\put(57,31){\line(-3,2){15}}
\put(55,24){ghi}

\put(32,0){\framebox(10,10){A}}
\put(52,0){\framebox(10,10){B}}
\put(37,12){\line(0,1){26}}
\put(37,12){\line(2,1){15}}
\put(57,12){\line(0,2){6}}
\end{picture}

\end{center}
\caption{\label{latexpic2}A diagram composed of circles, lines and boxes.}
\end{figure}



\section{Adding more complicated graphics}

The use of \LaTeX\ format can be tedious and it is often better to use
encapsulated postscript to represent complicated graphics.
Figure~\ref{epsfig} and ~\ref{xfig} on page \pageref{xfig} are
examples. The second figure was drawn using {\tt xfig} and exported in
{\tt.eps} format. This is my recommended way of drawing all diagrams.


\begin{figure}[tbh]
\centerline{\epsfbox{figs/cuarms.eps}}
\caption{\label{epsfig}Example figure using encapsulated postscript}
\end{figure}

\begin{figure}[tbh]
\vspace{4in}
\caption{\label{pastedfig}Example figure where a picture can be pasted in}
\end{figure}


\begin{figure}[tbh]
\centerline{\epsfbox{figs/diagram.eps}}
\caption{\label{xfig}Example diagram drawn using {\tt xfig}}
\end{figure}




\cleardoublepage
\chapter{Evaluation}

\section{Printing and binding}

If you have access to a laser printer that can print on two sides, you
can use it to print two copies of your dissertation and then get them
bound by the Computer Laboratory Bookshop. Otherwise, print your
dissertation single sided and get the Bookshop to copy and bind it double
sided.


Better printing quality can sometimes be obtained by giving the
Bookshop an MSDOS 1.44~Mbyte 3.5" floppy disc containing the
Postscript form of your dissertation. If the file is too large a
compressed version with {\tt zip} but not {\tt gnuzip} nor {\tt
compress} is acceptable. However they prefer the uncompressed form if
possible. From my experience I do not recommend this method.

\subsection{Things to note}

\begin{itemize}
\item Ensure that there are the correct number of blank pages inserted
so that each double sided page has a front and a back.  So, for
example, the title page must be followed by an absolutely blank page
(not even a page number).

\item Submitted postscript introduces more potential problems.
Therefore you must either allow two iterations of the binding process
(once in a digital form, falling back to a second, paper, submission if
necessary) or submit both paper and electronic versions.

\item There may be unexpected problems with fonts.

\end{itemize}

\section{Further information}

See the Computer Lab's world wide web pages at URL:

{\tt http://www.cl.cam.ac.uk/TeXdoc/TeXdocs.html}


\cleardoublepage
\chapter{Conclusion}

I hope that this rough guide to writing a dissertation is \LaTeX\ has
been helpful and saved you time.




\cleardoublepage

%%%%%%%%%%%%%%%%%%%%%%%%%%%%%%%%%%%%%%%%%%%%%%%%%%%%%%%%%%%%%%%%%%%%%
% the bibliography

\addcontentsline{toc}{chapter}{Bibliography}
\bibliography{refs}
\cleardoublepage

%%%%%%%%%%%%%%%%%%%%%%%%%%%%%%%%%%%%%%%%%%%%%%%%%%%%%%%%%%%%%%%%%%%%%
% the appendices
\appendix

\chapter{Latex source}

\section{diss.tex}
{\scriptsize\verbatiminput{diss.tex}}

\section{proposal.tex}
{\scriptsize\verbatiminput{proposal.tex}}

\section{propbody.tex}
{\scriptsize\verbatiminput{propbody.tex}}



\cleardoublepage

\chapter{Makefile}

\section{\label{makefile}Makefile}
{\scriptsize\verbatiminput{makefile.txt}}

\section{refs.bib}
{\scriptsize\verbatiminput{refs.bib}}


\cleardoublepage

\chapter{Project Proposal}

\vfil

\centerline{\Large Computer Science Project Proposal}
\vspace{0.4in}
\centerline{\Large PDB: A Distributed Database Based on Paxos}
\vspace{0.4in}
\centerline{\large Charlie Shepherd, Churchill College}
\vspace{0.3in}
\centerline{\large Originator: Charlie Shepherd}
\vspace{0.3in}
\centerline{\large 18$^{th}$ October 2012}

\vfil


\noindent
{\bf Project Supervisor:} Stephen Cross
\vspace{0.2in}

\noindent
{\bf Director of Studies:} Dr John Fawcett
\vspace{0.2in}
\noindent

\noindent
{\bf Project Overseers:} Dr~A.~Madhavapeddy \& Dr~M.~Kuhn


% Main document

\section*{Introduction, The Problem To Be Addressed}

{\em Paxos} is a protocol for achieving distributed consensus, developed by Leslie Lamport in 1991.

The motivation for Paxos as a protocol is that it is capable of handling failures that other
consensus protocols cannot. {\em Two Phase Commit} (2PC) and {\em Three Phase Commit} (3PC) are two common
protocols that can be used to ensure atomic commits in a distributed system.

2PC works by having a co-ordinator node contact every node and send a proposal message. Each node
must then either respond with a commit or abort message. However, 2PC suffers from several
problems, mainly that it is a blocking protocol. This means that if the co-ordinator fails, and
then a node subsequently fails, the network will deadlock, as 2PC is not able to recover from that
failure situation.

3PC is an extension to 2PC which endeavours to fix this limitation, at the expense of greater
latency, by adding a third roundtrip to confirm the commit to all nodes. This means that the
protocol is asynchronous, and that node failures cannot block the protocol or cause it to fail.
However, it still has its own limitations, in particular, in the event of a network partition. If
the network is partitioned so that in one partition all nodes vote ``commit'' and in the other all
nodes vote ``abort'' both partitions will initiate recovery, and when the network merges again the
system will be in an inconsistent state. This is the limitation that Paxos was intended to solve.

My project will be to design and implement a distributed database, built on the Paxos protocol.


\section*{Starting Point}

My starting point will be to study the Paxos protocol, as well as research distributed databases.
From there I will develop a library implementing Paxos, along with unit tests. I will then design
and implement a distributed database on the Paxos library. The challenge will be to implement a
complex distributed protocol and then utilise it for a database, as these are both areas I have
little experience of.

\section*{Resources Required}

I will mainly do my project on a virtual machine which is running on my own personal laptop.
The source code will be committed to a Git repository, which will be pushed to Bitbucket and my
own personal host. The virtual machine contents will be backed up on an external HDD for quick
recovery, although the git repository will be adequate for restoring my project if the system I am
developing it on fails.
I require no other special resources.

\section*{Work to be done}

The project breaks down into the following sub-projects:

\begin{enumerate}

\item A study of distributed algorithms and the Paxos protocol

\item A study of distributed databases

\item Implementing the Paxos protocol

\item Designing the distributed database

\item Implementing the distributed database

\item Evaluating the performance of the database

\end{enumerate}

\section*{Success Criteria for the Main Result}

In order for the project to be a success, the following must be true:

\begin{enumerate}

\item The project must correctly implement the Paxos protocol.
	The library must be capable of forming a running network,
	in particular dynamic leader election,
	as well as achieving consensus on a key/value store across the network.

\item The database must implement a subset of SQL, specifically:
\begin{enumerate}
\item A single table with a static name
\item SELECT/INSERT
\item WHERE
\item GROUP BY
\item ORDER BY
\item Aggregation
\end{enumerate}

\item The database must have all ACID properties, that is:
\begin{enumerate}
\item Atomic
\item Consistent
\item Isolated
\item Durable
\end{enumerate}

\end{enumerate}

\section*{Evaluation Topics}

There are several potential evaluation metrics for the project.

One major metric is transaction latency - the time for a transaction to be committed to the system. This
can be evaluated in a number of difference circumstances, including simulated node failure, leader
failure and network partition, and the results analysed to see how the system handles performance
under failure compared to normal conditions.

Another key metric is transaction throughput - the maximum number of transactions committed to the
network over a specified period of time. Again there are a number of different situations
throughput can be measured in, including load from one source, load from multiple sources and
load under failure.

I will also investigate the advantages of a distributed database over a normal single-server
database, particularly in terms of scalability. I will also consider how performance is affected
by the ratio of writing clients to reading clients.

\section*{Possible Extensions}

A clear possible extension is to investigate various different modifications to Paxos in order
to try to optimise the database for certain performance characteristics
(e.g. fast reads, but slow writes),
and to assess the usefulness and efficiency of these modifications.

Another possible extension is to investigate how the database performs when various ACID
properties are relaxed, and to measure and analyse how the performance gains compare with the
tradeoffs made.


\section*{Timetable: Workplan and Milestones to be achieved.}

\setlength\parindent{0pt}
\parskip = \baselineskip

Planned starting date is 19/10/2011.

\subsection*{Michaelmas Term}

{\bf 19/10/2012-01/11/2012} Research distributed algorithms and the Paxos protocol; design the
protocol implementation and library layout.

Milestone: A write up of the Paxos algorithm and a design document of the implementation.

{\bf 02/11/2012-15/11/2012} Begin the protocol implementation.

Milestone: Basic Paxos implementation.

{\bf 16/11/2012-29/11/2012} Finish implementation of Paxos library.

Milestone: Paxos implementation that can coordinate distributed leader election and achieve
consensus, including unit tests.

\subsection*{Christmas Vacation}

{\bf 30/11/2012-13/12/2012} Research distributed databases and design the database implementation.

Milestone: A write up of research on distributed databases and a design document for the database.

{\bf 14/12/2012-27/12/2012} Slack time/Revision/Holiday break.

{\bf 28/12/2012-10/01/2013} Prepare for progress report, start database implementation.

Milestone: Draft progress report, initial database implementation.

\subsection*{Lent Term}
{\bf 11/01/2013-24/01/2013} Write progress report, finish database implementation.

Milestone: Finished Progress report, fully functional database implementation.
Deadlines: Progress report deadline - 01/02/2013.

{\bf 25/01/2013-07/02/2013} Perform initial performance analysis on transaction, including
transaction latency and transaction throughput.

Milestone: Initial analysis data.

{\bf 08/02/2013-21/02/2013} Perform detailed performance analysis comparing distributed and
centralised servers, and on failing and partitioned networks.

Milestone: Analysis data on server models and on performance during failure.

{\bf 22/02/2013-07/03/2013} Investigate improvements to the protocol/implementation and their effect on
performance metrics.

Milestone: Improvements to protocol/implementation and revised performance data.

{\bf 08/03/2013-21/03/2013} Start dissertation.

Milestone: Draft Introduction and Preparation sections complete.

\subsection*{Easter Vacation}

{\bf 22/03/2013-04/04/2013} Finish writing up dissertation.

Milestone: Draft Implementation, Evaluation and Conclusion sections complete.

{\bf 05/04/2013-18/04/2013} Proof reading and then an early submission so as to concentrate on
examination revision.

Milestone: Finished dissertation.

\subsection*{Easter Term}
{\bf 19/04/2013-02/05/2013} Slack time/Revision/Holiday break.

{\bf 03/05/2013-16/05/2013} Slack time/Revision/Holiday break.

Deadlines: Official dissertation submission deadline - 17/05/2013.


\end{document}
